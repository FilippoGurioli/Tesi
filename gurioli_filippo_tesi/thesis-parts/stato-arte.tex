\chapter{Stato dell'arte}\label{chap:Letteratura}
Di seguito si riporteranno informazioni circa lo stato dell'arte riguardante la Extended Reality, in particolare, si toccherà il tema dello sviluppo di videogiochi in ambito 
Augmented Reality e Virtual Reality. Successivamente si analizzerà la letteratura riguardante il tema del multiplayer e come questo sia attualmente realizzato nei software comuni,
sottolineando l'importanza del web al giorno d'oggi.

\section{Extended Reality}\label{sec:XR}
    La Extended Reality è un campo molto vasto che comprende diverse tecnologie ed è in continua evoluzione. Talvolta non è possibile fornire una definizione univoca dei vari 
    concetti per via delle diverse visioni che gli esperti in materia hanno. Lo scopo della sezione seguente è quindi quello di chiarire i concetti fondamentali riguardanti
    questa tecnologia, esplorare le diverse accezioni e presentare alcuni esempi di utilizzo.
    \subsection{Definizione}\label{subsec:XRDef}
        Con il termine Extended Reality (abbreviato in XR) si intende un insieme di tecnologie che permettono di estendere la realtà, ovvero di aggiungere informazioni al mondo 
        reale. Sotto questa macro definizione rientrano le tecnologie di realtà aumentata (AR), realtà virtuale (VR) e realtà mista (MR). Per AR si intende la tecnologia che permette 
        di sovrapporre informazioni digitali al mondo reale, mentre per VR si intende la tecnologia che permette di immergere l'utente in un ambiente virtuale. La MR è una tecnologia
        che permette di conciliare il mondo reale con quello virtuale in modo totalmente trasparente per l'utente. Come ogni calcolatore, queste tecnologie prevedono dispositivi 
        di input e di output per l'interazione tra macchina e utente. Venendo completamente immersi in queste realtà virtuali, l'utilizzo di una tastiera o un mouse risulterebbe
        inefficace, si sono quindi progettati dispositivi ad hoc per l'interazione (input) e per la ricezione delle informazioni (output). Per quanto riguarda l'input, si possono
        utilizzare dei controller, dei guanti o il proprio corpo, mentre per l'output domina l'utilizzo di visori. Questi dispositivi sono in grado di rilevare i movimenti 
        dell'utente e di trasmetterli al sistema, che li elabora e li utilizza per modificare l'ambiente virtuale. Questo ambiente virtuale può essere visualizzato dall'utente 
        tramite un visore, che può essere un visore VR, un visore AR o un visore MR.
        \paragraph{Augmented Reality e Mixed Reality} sono due concetti molto simili, dei quali si trovano definizioni ambigue, talvolta discordanti. La definizione più comunemente
        accettata per Augmented Reality è quella fornita da Azuma\cite{Azuma1997}:
            \begin{quote}
                \textit{AR systems have the following three characteristics: (1) combine real and virtual, (2) are interactive in real time, and (3) are registered in 3D.}
            \end{quote}
            Si noti che non è specificato nessun tipo di dispositivo di output, potrebbero essere  AR anche dispositivi che forniscono informazioni tattili, gustative o olfattive,
            benchè la loro implementazione è di fatto un settore tutt'ora nascente (motivo per il quale nella sezione precedente non li si è menzionati). La definizione formale di 
            Mixed Reality è invece fornita da Milgram\cite{Milgram1994}:
            \begin{quote}
                \textit{A mixed reality (MR) system is one that combines real and virtual environments seamlessly.}
            \end{quote}
            Questa definizione è molto simile a quella di Azuma, ma non è specificato il fatto che l'ambiente virtuale debba essere tridimensionale. La definizione di Milgram non
            specifica, inoltre, che l'ambiente virtuale debba essere sovrapposto a quello reale, ma che i due ambienti debbano essere combinati in modo da risultare indistinguibili. 
            In quest'ottica si può quindi affermare che la AR è un caso particolare di MR, in cui l'ambiente virtuale è sovrapposto a quello reale.\\
            Il libro di Schmalstieg e Höllerer \textit{Augmented Reality: Principles and Practice}\cite{Schmalstieg2016} fornisce una schematizzazione della gerarchia tra AR e MR
            mostrata in figura\ref{fig:ARvsMR} che permette di comprendere meglio la relazione tra le due tecnologie. 
            \img{ARvsMR}{Gerarchia tra AR e MR - \textit{`The mixed reality continuum'}.}
            
    \subsection{Storiografia}\label{subsec:XRstoriografia}
        Come già accennato prima, il mondo della Extended Reality non si ferma al solo utilizzo di visori. Nella storia esistono esempi di applicazioni di AR e VR molto particolari 
        e interessanti. Si presenteranno di seguito alcuni di questi esempi, cercando di esplorare le scelte architetturali più interessanti.
        \paragraph{La spada di Damocle} (figura\ref{fig:SpadaDiDamocle}) è considerato  il primo prototipo di visore VR mai costruito nella storia e, come tale, ha diritto di essere
            citato come primo elemento di questa sezione. Il visore è stato costruito da Ivan Sutherland nel 1968 ed era pesante al punto da dover essere fissato al soffito per
            essere sorretto, da questo il suo nome. Il visore era in grado di mostrare all'utente un cubo tridimensionale, che poteva essere osservato da diverse angolazioni 
            semplicemente muovendo la testa, era infatti già presente, seppur in forma molto rudimentale, il concetto di \textit{head tracking}. Il visore era inoltre dotato di un 
            controller, che permetteva di interagire con l'ambiente virtuale. Questo prototipo è stato il primo passo verso la realizzazione di visori VR, ma non è stato
            mai commercializzato.
            \img{SpadaDiDamocle}{La spada di Damocle, primo prototipo di visore VR.}

        \paragraph{Sensorama} (figura\ref{fig:Sensorama}) è un dispositivo che permette di immergere l'utente in un
            ambiente virtuale, mostrando un video stereoscopico a 3 dimensioni e riproducendo suoni e odori. Il dispositivo è stato
            progettato per essere utilizzato in sale cinematografiche, in modo da coinvolgere lo spettatore con tutti e 5 i sensi.
            La macchina di Morton Heilig era però troppo costosa, e questo, unito alle dimensioni della macchina e alla scarsa qualità
            dei video, ha portato al fallimento del progetto.
            \img{Sensorama}{Sensorama, dispositivo fornito di: immagini stereo 3D, vibrazioni, vento, emettitore di odori e audio stereo.\cite{wikiSensorama}}

        \paragraph{IllumiRoom} è una tecnologia Microsoft che punta a creare un'esperienza AR estendendo la realtà di gioco
            normale, composta dallo schermo e dal controller, all'intera stanza. Questa tecnologia è stata presentata nel 2013
            e permette di proiettare sulle pareti della stanza in cui si gioca le immagini del gioco, in modo da estendere
            l'ambiente virtuale a quello reale. Questa tecnologia è stata sviluppata per la console Xbox One, ma non è mai
            stata commercializzata.\cite{Schmalstieg2016}
            \img{IllumiRoom.png}{IllumiRoom, prodotto capace di estendere l'ambiente di gioco all'intera stanza.}
        
        \paragraph{L'HTC Vive} è uno dei più famosi visori VR, è stato sviluppato da HTC e Valve Corporation ed è
            rilasciato nel 2016. L' \textit{Head Mounted Display} (HDM) è dotato di due schermi OLED da $1080\times1200$ pixel, 
            con un refresh rate di 90Hz\cite{Vive}. Questo visore è uno dei più conosciuti sul mercato, 
            e tra le altre cose, deve anche la sua fama alla tecnologia di tracciamento dei movimenti chiamata Lighthouse, che permette di tracciare
            i movimenti dell'utente in modo preciso e affidabile. Questa tecnologia è stata sviluppata dalla Valve
            e consiste di due stazioni base che emettono dei segnali infrarossi che vengono rilevati da sensori posti 
            sulle periferiche. Queste stazioni base vengono poste alle estremità della stanza, in modo tale da circoscrivere
            un'area entro la quale l'utente è libero di muoversi.
            \img{HTCVive.jpg}{HTC Vive, da sinistra a destra: stazioni base, controller e visore.}

        \paragraph{I dispositivi mobili} sono alla base della maggior parte delle applicazioni di AR. Questi dispositivi
            sono dotati di fotocamera, accelerometro, giroscopio, magnetometro e geolocalizzazione che permettono di rilevare i movimenti
            dell'utente e di orientare la fotocamera. Questi dispositivi sono inoltre dotati di schermo, che permette di
            visualizzare l'ambiente virtuale. I cellulari sono molto utilizzati per applicazioni di Augmented Reality, in quanto
            sono dispositivi che tutti possiedono e che sono in grado di fornire un'esperienza soddisfacente.
            Attualmente sul mercato è esploso il fenomeno degli \textit{AR advertising}, ovvero applicazioni che permettono
            di visualizzare i prodotti venduti dalle aziende in modo virtuale direttamente nello spazio reale. Queste applicazioni
            sono molto utilizzate per la vendita di mobili, in quanto permettono di proiettare il prodotto in modo realistico
            e di valutare se questo si adatta all'ambiente in cui si vuole inserire. Anche il settore videoludico non ha 
            perso l'occasione di sfruttare questa tecnologia, infatti sono state sviluppate applicazioni che permettono di
            visualizzare i personaggi dei propri giochi preferiti direttamente nella propria stanza. Un esempio di questo
            tipo di applicazioni è \textit{Pokémon GO}, gioco sviluppato da Niantic nel 2016, che permette di catturare i
            Pokémon direttamente nella propria città.
            \img{PokemonGo.png}{Pokémon GO, applicazione di AR per dispositivi mobili.}
            
        \paragraph{Hololens} è il dispositivo Microsoft rilasciato nel 2016 che ha portato il mondo delle tecnologie 
            AR e MR a un livello superiore. Questo HMD è dotato di un processore di fascia alta, una GPU per il calcolo
            di immagini ed una Holographic Processing Unit (HPU) che permette di elaborare i dati provenienti dai sensori
            in tempo reale. Il visore monta anche una telecamera di profondità, sensori avanzati per il tracciamento dei
            movimenti e un sistema di rilevamento del suono. Questo visore è in grado di proiettare immagini in modo
            realistico sulle lenti con display ottico 3D, in modo da sembrare che l'immagine sia davanti all'utente.
            Il visore ha un costo ancora troppo elevato per essere utilizzato da un utente comune, ma è stato utilizzato
            in ambito industriale per la progettazione di automobili e per la formazione di chirurghi. \\
            Tra le varie feature che il visore offre ci sono:
            \begin{itemize}
                \item \textbf{Spatial Mapping}: permette di mappare l'ambiente circostante e di rilevare gli oggetti
                    presenti, in modo da poterli utilizzare per interagire con l'ambiente virtuale.
                \item \textbf{Spatial Sound}: permette di rilevare la posizione dell'utente e di modificare il suono in base
                    ad essa.
                \item \textbf{Gaze Tracking}: permette di rilevare la direzione dello sguardo dell'utente.
                \item \textbf{Voice Recognition}: permette di rilevare i comandi vocali dell'utente.
                \item \textbf{Gesture Recognition}: permette di rilevare i movimenti delle mani dell'utente svincolando
                    l'utente dall'uso di controller e guanti (un ulteriore passo avanti verso la \textit{computazione ubiqua}
                    \footnote{Computazione ubiqua: modello post-desktop di interazione uomo-macchina, in cui l'elaborazione
                    delle informazioni è stata interamente integrata all'interno di oggetti e attività di tutti i giorni.\cite{ubiCompWiki}}).
            \end{itemize}
            In conclusione si noti che ogni sensore è completamente integrato nel visore, che non necessita quindi di
            dispositivi esterni rendendolo completamente wireless.
            \img{Hololens.png}{Hololens, visore AR/MR di Microsoft.}

    \subsection{Applicazioni}\label{subsec:XRapplicazioni}
        Le applicazioni di Extended Reality sono molteplici e vanno dai videogiochi, all'industria, alla medicina. In questa sezione si analizzeranno alcune applicazioni di AR e VR, 
        cercando di capire quali sono le scelte architetturali più interessanti.
        \paragraph{L'industria} sta utilizzando queste nuove tecnologie per migliorare la produttività e la qualità dei prodotti. Un esempio di questo utilizzo è quello già
            accennato di AR Ads per la vendita di mobili, vestiti e altre merci, di cui i principali esempi sono:
            \begin{itemize}
                \item American Apparel che ha sviluppato un'applicazione in grado di mostrare i vestiti in modo realistico sul proprio corpo;
                \item IKEA che ha sviluppato un'applicazione che permette di visualizzare i propri mobili direttamente nella propria casa;
                \item De Beers che ha sviluppato un'applicazione che permette di provare gli anelli direttamente sul proprio dito.
            \end{itemize} 
            Un altro esempio di utilizzo di AR in ambito industriale è quello di \textit{Remote Assistance}, ovvero l'assistenza remota. Questa tecnologia permette di fornire 
            assistenza a distanza, dando la possibilità di vedere ologrammi illustrativi direttamente nello spazio reale, facilitando il lavoro dei tecnici e permettendo di 
            ridurre i costi di formazione.\\

        \paragraph{Augmented Medicine} è il termine che è stato coniato per indicare l'utilizzo di dispositivi AR in ambito medico. Tra queste applicazioni si possono citare:
            \begin{itemize}
                \item \textbf{Formazione di chirurghi}: in quanto permette di visualizzare in modo realistico gli organi 
                    interni del paziente, permettendo di eseguire operazioni in modo virtuale e di simulare interventi chirurgici.
                \item \textbf{Progettazione di protesi}: in quanto permette di visualizzare in modo realistico le protesi 
                    direttamente sul paziente, permettendo di valutare se la protesi si adatta al corpo del paziente.
                \item \textbf{Vista in profondità}: l'AR ha permesso a chirurghi di vedere in tempo reale tumori, fratture e aneurismi direttamente sul paziente, in modo da poter 
                    agire in modo più preciso e sicuro.
            \end{itemize}
            \img{ARMedicine.png}{Esempio di applicazione di AR in ambito medico.}

        \paragraph{I videogiochi} rappresentano il settore che più di tutti ha sfruttato le potenzialità di queste tecnologie. I videogiochi sono stati sviluppati per essere giocati 
            su schermi bidimensionali, ma con l'avvento di queste nuove tecnologie si è aperto un nuovo mondo di possibilità. Sin dagli inizi della realtà virtuale si possono notare
            i primi prodotti di successo come \textit{Virtuality}, \textit{BeatSaber} e \textit{Half-Life: Alyx}. Queste applicazioni hanno sfruttato le potenzialità di queste
            tecnologie per creare esperienze di gioco uniche, coinvolgenti e realistiche.\\
            Per avere una panoramica più dettagliata di queste applicazioni si rimanda alla sezione successiva (\ref{sec:Videogiochi}) dove verranno spiegati in dettaglio
            il significato, la storia e le applicazioni più famose dei videogames.

\section{Videogiochi}\label{sec:Videogiochi}
    In questa sezione si parlerà di videogiochi, in particolare si darà una definizione formale, si analizzerà la storia di questo settore discutendone le applicazioni più famose e 
    si parlerà delle varie tipologie di videogiochi.
    
    \subsection{Definizione}\label{subsec:VideogiochiDef}
        Da quando il settore videoludico ha fatto la sua ascesa il termine `gioco' non ha più avuto lo stesso significato. La parola `gioco' viene utilizzata per indicare un'attività 
        ludica, che ha come scopo il divertimento, la spensieratezza e il passatempo. Tra le varie attività ludiche si possono annoverare i giochi da tavolo (Monopoli, Risiko), i giochi di
        carte (scala quaranta, bridge), i giochi sportivi (frisbee, calcio) o i semplici giochi per bambini tradizionali (strega comanda colore, uno due tre stella). È in questo contesto che
        il videogioco prende piede, mantenendo l'accezione comune di gioco come attività ludica con un regolamento e un'ambientazione, ma aggiungendo un elemento che lo distingue da tutti gli
        altri giochi: l'interazione con un dispositivo elettronico. \\
        Il termine videogioco significa `gioco gestito da un dispositivo elettronico'. Sono videogiochi quindi i conosciutissimi giochi online, i giochi per cellulare, i giochi per console 
        e i giochi per computer, ma anche i cabinati e le slot machine soddisfano questa definizione in quanto non solo hanno un software che gestisce il gioco, ma anche un hardware dedicato.\\
        Il termine `videogioco' è stato coniato da Ralph Baer nel 1966, utilizzato dall'ingegnere per descrivere il proprio progetto di gioco elettronico (di fatto la prima console della storia).
        Ai giorni d'oggi questo termine ha un significato quasi completamente distaccato dall'accezione generale di gioco, esistono comunità intere che fanno dei videogames una professione,
        esistono competizioni internazionali e campionati mondiali e ci sono anche videogiochi che hanno un budget di produzione maggiore di quello di un film. In questo contesto anche l'utente
        medio si approccia al gioco in modo più serio, non più come un semplice passatempo, ma come un'esperienza da vivere e da condividere con gli altri. Quello che si può definire come
        un vero e proprio fenomeno culturale ha stravolto molti paradigmi: il tempo che si dedica a conoscere il gioco, sia direttamente giocandolo, che indiriettamente informandosi su di esso,
        l'accezione che ormai ha perso il sinonimo di passatempo, il luogo in cui si gioca ed il modo in cui si gioca.\\ 

        \subsection{Storiografia}\label{subsec:VideogiochiStoriografia}
        La storia dei videogiochi è molto lunga e complessa, in questa sezione si cercherà di riassumere i punti salienti di questa storia passando per gli esempi più famosi e le
        innovazioni più importanti.

        \paragraph{Pong} è il videogioco per antonomasia, precursore della storia dei video games. Questo gioco è stato sviluppato da Atari nel 1972 ed è stato il primo gioco arcade
            di successo. Il gioco consiste in una simulazione di ping pong in cui due giocatori si sfidano a colpi di palla. In questo semplice esempio si concretizza già il primo 
            multiplayer, Pong è infatti giocabile sia in singolo, contro il computer, che in due giocatori, mostrando sin da subito che un gioco ha più valore se giocato in compagnia. 
            Bisogna ricordare Pong anche come uno dei primi \textit{coin-op} della storia, ovvero uno dei primi giochi arcade che richiedevano l'inserimento di una moneta per poter 
            essere giocati.
            \img{Pong2.png}{Pong, il primo videogioco arcade di successo.}

        \paragraph{Pac-Man} è il gioco ideato da Toru Iwatani, prodotto dalla Namco, che ha appassionato molti nelle salegiochi degli anni '80. Il gioco consiste di un personaggio giallo,
            comandato dal giocatore, che ha lo scopo di mangiare tutti i puntini presenti nel labirinto, evitando di essere mangiato dai fantasmi. Esistono anche delle `pillole speciali'
            ai lati del labirinto che permettono al giocatore di diventare immune ai fantasmi per 10 secondi. Infine, se il giocatore riesce a mangiare tutti i puntini del labirinto,
            viene mostrato un intermezzo, per poi far ricominciare il gioco da capo (mantenendo il punteggio). Il gioco deve il suo successo a diversi fattori tra cui: 
            essere uno dei pochi giochi non violenti in un mercato dominato da giochi di guerra, avere dei comandi semplici e intuitivi e avere un gameplay basato più sulla tattica 
            che sui riflessi\cite{Uston1982}.
            \img{PacMan.png}{Pac-Man, il gioco arcade più famoso di tutti i tempi.}

        \paragraph{Super Mario Bros.} è la punta di diamante della Nintendo che seppur non sia il primo Mario mai creato, è stato il più celebre. Questo gioco ha gettato le fondamenta
            per tutti i giochi platform che si sarebbero creati negli anni successivi. Questo consiste in un platform a scorrimento orizzontale in cui il giocatore controlla Mario, 
            un idraulico che deve salvare la principessa Peach dal malvagio Bowser. È presente anche la possibilità di giocare in cooperativa con un altro giocatore, che controllerà
            Luigi, il fratello di Mario. Il video game di Miyamoto ha portato un vento di innovazione nel settore, portando a fama mondiale il concetto livelli non lineari, ovvero
            livelli in cui il giocatore può scegliere il percorso da seguire, e il concetto di power-up, ovvero oggetti che permettono al giocatore di ottenere abilità speciali.\cite{ryan2011}
            \img{SuperMarioBros.jpg}{Super Mario Bros., il gioco che ha rivoluzionato il mondo dei platform.}
            
        \paragraph{Doom}
        
        \paragraph{Minecraft}

    \subsection{Categorie}\label{subsec:VideogiochiCategorie}
        \paragraph{Avventura}

        \paragraph{Sparatutto}
        
        \paragraph{Di ruolo}

        \paragraph{Di strategia}

        \paragraph{Online}

\section{Multiplayer}\label{sec:Multiplayer}
    \subsection{Definizione}\label{subsec:MultiDef}
    \subsection{Tipologie}\label{subsec:MultiTipologie}
    Qui si parlerà delle varie tipologie di multiplayer, facendo riferimento anche al libro Multiplayer Game Programming.
    \subsection{Applicazioni}\label{subsec:CoopApplicazioni}
        \paragraph{Al giorno d'oggi}\label{par:MultiOggi}
        \paragraph{Web}\label{par:CoopWeb}
