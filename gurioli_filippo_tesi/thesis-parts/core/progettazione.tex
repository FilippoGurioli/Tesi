\section{Progettazione dettagliata}\label{sec:progettazione}
Qui verranno trattati dettagli implementativi come BaseModel e BaseView;
i vari scambi di eventi e messaggi tra i vari componenti etc

Cose da dire:
\begin{itemize}
\item Si è strutturata una rete di \texttt{Croquet.Model} che distribuisse ogni funzionalità del dominio ad un modello diverso, in modo da tenere separati i compiti e mantenere una
    gerarchia ordinata e intuitiva.

\item Lungo tutta la \textit{codebase} si è deciso di non utilizzare ereditarietà, preferendo la composizione in quanto ritenuta più flessibile e meno vincolante. La struttura si può
raffigurare comunque in una relazione padre-figlio in cui però non è il figlio a estendere il padre, bensì il padre a creare e contenere il figlio.\\

\item Definiti questi princìpi cardine si sono quindi costruite le fondamenta su cui basare il video game. La classe più importante, da cui poi si sviluppa tutta la struttura, è 
\texttt{GameModel}. Questa classe di fatto non gestisce una struttura dati ma fa da contenitore per tutti i modelli del gioco. Qui si trovano i riferimenti ai modelli dei giocatori,
del turno e del campo di battaglia.\\
Altra funzionalità importante per questa classe è di gestire le connessioni e i ruoli. All'avvio, in base al numero di partecipanti già presenti, questa classe avrà il compito di 
assegnare un ruolo al nuovo utente connesso scegliendo tra \textit{player 1}, \textit{player 2} e \textit{spettatore}. Inoltre, se un utente dovesse disconnettersi, questa classe
dovrà gestire una sua possibile riconnessione come anche prevedere una sequenza di terminazione nel caso in cui l'utente non dovesse riconnettersi.\\
Nella controparte \texttt{GameView} si possono trovare gli stessi riferimenti alle view corrispondenti dei modelli citati. Si noti che, dato che il model non contiene una struttura
dati, la \texttt{GameView} non crea alcun componente visibile all'utente, mantenendo coerenza con il principio di specularità tra model e view.\\

\item Il back-end nel progetto è rappresentato da un insieme di \texttt{Croquet.Model} che si occupano di gestire i dati e di fornire le funzionalità base per interagirci. Spesso vengono
anche utilizzate classi standard JavaScript contenute all'interno dei model che fungano da supporto per gestire le strutture dati. Si è deciso di non utilizzare ereditarietà 
all'interno del progetto, preferendo la composizione, in quanto si è ritenuto che non fosse necessario avere una gerarchia di classi. 
La prima classe model creata, nonchè quella passata 
\end{itemize}