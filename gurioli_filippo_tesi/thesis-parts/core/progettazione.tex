\section{Progettazione dettagliata}\label{sec:progettazione}
In quest'ultima sezione si affronteranno le strutture del progetto nello specifico, analizzando i requisiti del progetto e come questi siano stati implementati nel concreto.\\
\newline
Prima di iniziare con la descrizione, si vuole esporre dei punti chiave comuni lungo tutto il progetto.\\
Seguendo la logica del \textit{single-responsibility principle}\footnote{Single responsibility principle: principio di progettazione software che afferma che ogni modulo o classe deve
essere responsabile di una singola funzionalità.}, si è strutturata una rete di \texttt{Croquet.Model} che distribuisse ogni funzionalità del dominio ad un modello diverso, in modo da 
tenere separati i compiti e mantenere una gerarchia ordinata e intuitiva.\\
Altro tassello importante è la decisione di non utilizzare ereditarietà, preferendo la composizione in quanto ritenuta più flessibile e meno vincolante. La struttura si può
raffigurare comunque in una relazione padre-figlio in cui però non è il figlio a estendere il padre, bensì il padre a creare e contenere il figlio.\\
In ultima istanza si sono create delle classi JavaScript ausiliarie all'architettura di Croquet che fungono da supporto per la gestione dei dati. Queste classi sono state create per 
racchiudere al loro interno le caratteristiche più importanti dei concetti da modellare e quindi facilitare la manutenzione e migliorare la leggibilità del codice.\\

\subsection{Visualizzazione degli ologrammi personali e condivisi in tempo reale}
Si definisce ologramma condiviso un ologramma che è visibile da tutti gli utenti connessi alla sessione. Al contrario, un ologramma personale è un oggetto di scena che può essere diverso
da utente a utente e talvolta anche non esserci.\\
Per realizzare un ologramma condiviso basterà creare nella view un oggetto di scena con coordinate assolute. Visto che tutti gli utenti eseguono lo stesso codice, quando
dovranno istanziare l'ologramma condiviso lo faranno tutti allo stesso modo. Per il medesimo motivo, quando il model lancia un evento per comunicare l'aggiornamento dello stato, tutte le 
view lo riceveranno aggiornando l'ologramma alla stessa maniera.\\
Per creare un ologramma personale si è ricorsi all'istanziamento degli oggetti in modo condizionale. Questo significa che, in base a certe condizioni, vengono creati oggetti di scena 
diversi. Nel caso particolare del videogioco, in base al ruolo che ricopre l'utente si istanzieranno viste diverse. Ad esempio, se l'utente è un player ed è il suo turno verrà mostrato un
\textit{near menù}, ovvero un menu capace di seguire l'utente (nativo di MRTK e integrato in BabylonJS), per procedere di fase, al contrario non verrà mostrato nulla. Si fornisce il 
codice~\ref{lst:ologrammi.js} come esempio. 
\code{ologrammi.js}{Esempio di istanziazione di ologrammi personali e condivisi.}

\subsection{Interazioni del giocatore}
Per tutte le interazioni del giocatore lungo tutto il corso del progetto si è seguito sempre lo stesso pattern:
\begin{enumerate}
    \item l'utente interagisce con un oggetto di scena;
    \item l'oggetto di scena lancia un evento;
    \item il model riceve l'evento e aggiorna lo stato;
    \item il model lancia un evento per comunicare l'aggiornamento dello stato;
    \item la view riceve l'evento e aggiorna la scena.
\end{enumerate}
Si prenda d'esempio l'interazione con la mano del giocatore. L'utente interagisce con una carta tra quelle presenti nello spazio personale, la carta lancia un evento \textit{`playCard'}
che viene intercettato del model corrispondente \texttt{HandModel}. Questo verifica che ci siano le condizioni adatte per poter giocare la carta e, se così fosse, aggiorna lo stato
rimuovendo la carta dalla mano e aggiungendola al campo di battaglia. Il model quindi lancia due eventi: uno per rimuovere la carta dalla mano ed uno per aggiungere la carta al
campo di battaglia. La view \texttt{HandView} riceve l'evento \textit{`removeCard'} e rimuove la carta dallo spazio personale, mentre la view \texttt{BattlefieldView} riceve l'evento
\textit{`placeCard'} e crea la carta nel campo di battaglia. Si lascia il codice~\ref{lst:interazioni.js} come esempio.
\code{interazioni.js}{Esempio di interazione tra utente e oggetto di scena e conseguente aggiornamento dello stato.}

\subsection{Gestione delle connessioni}
Nel progetto si è dato un peso rilevante a tutta la parte di gestione delle connessioni e disconnessioni degli utenti. Si sono create infrastrutture apposite che sapessero controllare
e mantenere in uno stato coerente le connessioni di tutti i giocatori.\\
Questo compito lo si è affidato alle classi \texttt{GameModel} e \texttt{GameView}, figlie dirette delle \textit{root class}, ovvero le prime classi create nel progetto.
Queste classi hanno il compito di gestire le comunicazioni e di istanziare tutti i modelli e le viste del gioco. Si sono dovute gestire tutte le casistiche possibili, come la 
connessione di un nuovo giocatore, la disconnessione temporanea di un giocatore, la riconnessione di un giocatore e la disconnessione definitiva di un giocatore. 
Per dare una panoramica generale di come si è risolto il problema si lascia il codice~\ref{lst:connessioni.js} d'esempio. Nel listato è stata omessa la classe \texttt{GameView} per 
brevità, preferendo una spiegazione discorsiva delle funzionalità.\\
La classe \texttt{GameView} si occupa di istanziare la scena di gioco e di posizionare la telecamera dell'utente in base al ruolo che gli è stato assegnato.
In aggiunta, alla ricezione dell'evento `opponent-left', interromperà il normale flusso di gioco, mostrando agli utenti rimasti una schermata d'attesa. Scatenando l'evento 
`opponent-recover' invece, farà riprendere la scena da dove era stata interrotta. Infine, all'avvento del `game-over', la \texttt{GameView} distruggerà la scena e ricreerà la scena
di base.\\
\code{connessioni.js}{Infrastruttura per la gestione delle connessioni.}
Un altro aspetto importante che è stato gestito è la possibilità di riavviare la simulazione senza dover ricaricare la pagina. Se tutte le classi si autodistruggessero alla fine del
gioco, l'utente si ritroverebbe con una schermata vuota e dovrebbe ricaricare la pagina per avviare una nuova sessione. Per aggirare il problema si è deciso di cambiare il 
comportamento che hanno le \textit{root class} al termine della partita, facendo sì che ricreino la scena (e anche i dati) da zero come mostrato nel codice~\ref{lst:root.js}.
\code{root.js}{Creazione e riavvio della simulazione.}

%________________________________________________________________________________________________________________________________________________________________________________________
\subsection{Root}\label{subsec:root}
Il \texttt{RootModel} e la \texttt{RootView} sono le prime classi create nel progetto. Queste classi sono state create con la funzione di modellare una scena di base da cui far
partitre l'utente, per creare una separazione tra il collegamento alla sessione e l'inzio del gioco vero e proprio. L'obiettivo era infatti quello di rendere la gestione delle
connessioni più semplice e logico.\\
Il \texttt{RootModel} nello specifico tiene traccia di tutte le view che si collegano alla sessione, in modo da creare un'istanza del gioco all'avvento della prima view collegata e 
distruggere il gioco alla disconnessione dell'utlima view. Fornisce anche il metodo \texttt{restart} che consente di riavviare la simulazione in caso di fine gioco ma ancora utenti 
connessi.\\
La classe \texttt{RootView} si occupa di creare la scena di base, ovvero quella che l'utente vede prima di iniziare il gioco. Questa consiste semplicemente di un bottone che, se 
premuto, avvia la simulazione vera e propria. Seppur possa sembrare abbasta inutile, la parte più importante di questa classe risiede nel \textit{dietro alle quinte} della scena. 
Qui infatti si crea e avvia un \textit{rendering-loop}, ovvero un ciclo infinito che aggiorna la scena ad ogni frame, l'\textit{engine}, la scena, la telecamera, la luce e tutta
una serie di funzionalità dedite al corretto avvio di una scena XR (se è stato rilevato un dispositivo affine).\\
Queste due classi sono le uniche in tutta la \textit{codebase} a non autodistruggersi all'avvento del `game over'. Se così non fosse, al termine della partita l'utente si ritroverebbe
con una schermata vuota e dovrebbe ricaricare la pagina per avviare una nuova sessione. Per aggirare il problema si è deciso di cambiare il comportamento che hanno queste classi al 
termine della partita, facendo sì che ricreino la scena (e anche i dati) da zero.\\

\subsection{Game}\label{subsec:game}
Le classi \texttt{GameModel} e \texttt{GameView} sono le classi principali del progetto e fungono da raccordo tra le tutte le altre. Il loro principale compito è quello di creare
tutti i tasselli che compongono il videogioco e svolgere funzionalità ad alto livello come l'assegnazione del ruolo e la gestione delle connessioni.\\
Il \texttt{GameModel} ha il compito di assegnare un ruolo all'utente che si connette alla sessione. Questo ruolo può essere di \textit{player 1}, \textit{player 2} o \textit{spettatore} in 
base al numero di utenti già connessi. Inoltre, se un utente dovesse disconnettersi, questa classe gestisce una sua possibile riconnessione come anche una sequenza di terminazione nel
caso in cui l'utente non dovesse riconnettersi in breve. Come già detto il \texttt{GameModel} ha anche il compito di istanziare tutti i modelli del gioco, ovvero: \texttt{Player},
\texttt{LifePoints}, \texttt{Hand}, \texttt{Turn} e \texttt{Battlefield}.\\
La \texttt{GameView} non crea nessun tipo di oggetto di scena bensì si limita a riposizionare la telecamera dell'utente in base al ruolo che gli è stato assegnato. Successivamente istanzia
tutte le view corrispondenti ai model citati in precedenza.\\

\subsection{Player}\label{subsec:player}
Le classi \texttt{PlayerModel} e \texttt{PlayerView}, similmente alle classi `Game', sono classi che fanno da centro per tutte le altre che gestiscono un aspetto particolare del giocatore.
Queste classi non hanno nessuna funzione di business, il loro impiego è stato voluto per mantenere organizzato il codice. Da queste classi si sviluppano quindi \texttt{LifePoints} e
\texttt{Hand}.\\

\subsection{LifePoints}\label{subsec:lifepoints}
Le classi che modellano i \textit{lifepoints}, ovvero i punti vita dei giocatori, sono \texttt{LifePointsModel} e \texttt{LifePointsView}.\\
Il \texttt{LifePointsModel} sfrutta una classe ausiliaria chiamata \texttt{LifePoints} che racchiude tutte le funzioni e informazioni necessarie per gestire i punti vita di un giocatore.
Seguendo il regolamento, ogni giocatore inizia con 8000 punti vita, può guadagnarne all'infinito e perderne fino ad arrivare a 0. Il \texttt{LifePointsModel} mappa tutta l'API creata
in \texttt{LifePoints} e la rende disponibile alle view con la \textit{feature} in più di lanciare l'evento `game-over' quando un giocatore perde tutti i punti vita. Dato che possono
esserci vari modi per perdere e guadagnare punti vita si è deciso di mappare questi metodi ad eventi correlati.\\
La \texttt{LifePointsView} si occupa di creare un \textit{holographic slate}, ovvero un pannello nativo dell'MRTK framework integrato in BabylonJS, che mostra i punti vita dei giocatori.
Questo pannello sarà personale di ogni giocatore e mostrerà a sinistra i punti vita del proprietario mentre a destra quelli dell'avversario. Si è anche prevista la possibilità che il
pannello fosse ancorato al braccio nel qual caso l'\textit{hand tracking feature} fosse disponibile.\\

\subsection{Card}\label{subsec:card}
La struttura che modella le carte del gioco è un file JSON che contiene tutte le informazioni necessarie per creare una carta. Si è deciso di uscire dalla classica struttura model-view
in questo caso per il fatto che, per come era stato modellato il dominio, le carte non presentassero alcuna funzione di business. Il modo per rappresentare meglio una carta è quindi
una struttura che ti permetta di immagazzinare tutte le informazioni che la compongono, in questo caso un file JSON. Le informazioni salvate sono: ID (univoco), nome, tipo e descrizione,
e attacco e difesa solo per le carte di tipo mostro. Questa deve essere vista come una struttura provvisoria al solo scopo dimostrativo dell'applicazione, in una versione commerciale
questa struttura dati dovrebbe essere sostituita da un database.\\

\subsection{Hand}\label{subsec:hand}
Le classi che gestiscono la mano del giocatore (intesa come insieme di carte visibili) prendono il nome di \texttt{HandModel} e \texttt{HandView}.\\
Dato che una carta è identificata univocamente dal suo ID si è deciso di gestire la mano come una lista di ID. Questa lista è contenuta e gestita all'interno del 
\texttt{HandModel} che fornisce anche i metodi per aggiungere, rimuovere e giocare carte dalla mano. In quest'ultimo caso si avvia una procedura speciale che comprende la rimozione
della carta dalla mano e la creazione della stessa all'interno del campo di battaglia.\\
La \texttt{HandView} per ogni ID presente nella lista all'interno del corrispettivo modello, genera nello spazio personale del giocatore la carta corrispondente. Queste carte sono 
afferrabili e possono essere giocate sul campo di battaglia.\\

\subsection{Turn}\label{subsec:turn}
Per la gestione del concetto di turno si è applicato lo stesso pattern dei \textit{life points}, ovvero una classe ausiliaria chiamata \texttt{Turn} che contiene tutte le funzioni e
informazioni necessarie per gestire il turno di un giocatore, una classe \texttt{TurnModel} che mappa l'API di \texttt{Turn} e una classe \texttt{TurnView} che crea un
\textit{holographic slate} che mostra al giocatore il turno e la fase corrente. Altre funzionalità importanti sono la possibilità di avanzare di fase e, se si è all'ultima fase, di
passare il turno. Per permettere all'utente di avanzare di fase si è disposto un \textit{near menu}, ovvero un menu capace di seguire l'utente (nativo di MRTK e integrato in 
BabylonJS) che, se interagito, fa passare alla fase successiva.

\subsection{Battlefield}\label{subsec:battlefield}
Il campo di battaglia è il punto focale in cui l'interazione tra i due utenti si concretizza. Tutte le \textit{mesh} create all'interno di queste classi dovranno essere condivise
tra tutti gli utenti connessi alla sessione. Per supportare lo sviluppo del concetto di campo da gioco si sono utilizzate le classi \texttt{Battlefield}, \texttt{BattlefieldModel} e
\texttt{BattlefieldView}.\\
Nella classe JavaScript \texttt{Battlefield} viene inserita tutta la struttura dati che serve a gestire il campo di battaglia. Sono presenti quattro liste: una per i mostri e una per
le carte magia e trappola, entrambe ripetute per ogni giocatore. La classe, infine, fornisce metodi per inserire e rimuovere carte dalle liste sopra menzionate.\\
La classe \texttt{BattlefieldModel}, oltre che incapsulare la classe \texttt{Battlefield} mappando la sua API, fornisce anche metodi unici per l'accesso alla struttura dati affinchè
si renda più facile ma anche più sicura la lettura dei dati condivisi.\\
Per ultima, la classe \texttt{BattlefieldView} si occupa di mostrare agli utenti il campo e le carte giocate. In un primo momento crea solo un piano, corrispondente al campo da gioco. 
All'avvento dell'interazione utente per il posizionamento di una carta, la \texttt{BattlefieldView} crea la carta in corrispondenza della sua posizione nella struttura
dati nel model.\\
\newline
Si vuole terminare questo capitolo dicendo che l'infrastruttura creata con le classi esposte non è da ritenersi conclusa. Come già specificato, data l'ambizione della tesi di 
trattare argomenti nuovi per il corso di laurea, non si è potuta esplorare la realizzazione di tutti i requisiti del progetto. Ciò nonostante si è cercato di raggiungere un 
prodotto minimo funzionante che racchiudesse le dinamiche principali del gioco e che, per quanto semplice, potesse essere giocato dall'inizio alla fine.
