\section{Analisi dei requisiti}\label{sec:Analisi}
    L’obiettivo del progetto è creare un’ambiente di realtà aumentata condivisa
    in cui l’utente possa giocare contro un altro al gioco di carte Yu-Gi-Oh. L’esperienza che il giocatore proverà dovrà essere quanto più simile alla versione
    proposta nella serie animata omonima.
    \newline \newline
    Al momento dell’avvio l’utente dovrà affrontare un duello contro un’altra persona a Yu-Gi-Oh. Per la decisione del regolamento da seguire si è optato per
    una versione semplificata del gioco. Il giocatore potrà giocare carte mostro che
    rappresentano delle truppe schierate dalla parte del possessore. Queste truppe
    potranno quindi attaccare l’avversario per ridurne i punti vita. Saranno presenti anche carte magia e trappola che, tra i vari effetti, potranno modificare
    i punti vita, l’ambiete di gioco in cui gli utenti giocano o anche l’attacco e
    la difesa dei mostri propri e avversari. L’obiettivo del gioco consiste quindi
    nell’azzerare i punti vita dell’avversario, che comporterà la conclusione della
    simulazione.

    \subsection{Requisiti funzionali}\label{subsec:requisitiFunzionali}
        \begin{itemize}
            \item Il giocatore sarà in grado di vedere gli ologrammi personali e condivisi in tempo reale;
            \item il giocatore potrà interagire con un mazzo di carte virtuale pescando la prima carta;
            \item il giocatore potrà posizionare le carte che ha in mano sul campo e di
            conseguenza far apparire la corrispondente carta nello spazio di gioco
            condiviso;
            \item il giocatore potrà avanzare di fase, come ordinare l'attacco di un mostro tramite un menù apposito;
            \item ad ogni danno (o cura) inflitto (o subito) verrà visualizzato un ologramma condiviso che mostra i punti vita rimanenti.
        \end{itemize}
    \subsection{Requisiti non funzionali}\label{subsec:requisitiNonFunzionali}
        \begin{itemize}
            \item L'applicazione dovrà sfruttare la tecnologia WebXR per rendere fruibile, tramite un qualsiasi browser compatibile, l'esperienza di gioco.
        \end{itemize}