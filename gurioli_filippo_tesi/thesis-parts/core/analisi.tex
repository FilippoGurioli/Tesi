\section{Analisi dei requisiti}\label{sec:Analisi}
    Questa sezione si pone l'obiettivo di introdurre il lettore agli argomenti trattati nelle sezioni successive, dove vengono affrontati dettagli tecnici del progetto. Qui si esporranno
    in maniera generale l'idea alla base del progetto, le sue sfide principali e il traguardo che si vuole conseguire.\\
    \newline
    Il software che si vuole sviluppare consiste in una rivisitazione del gioco \textit{Yu-Gi-Oh!} trasportato in realtà aumentata. \textit{Yu-Gi-Oh!} è un gioco di 
    carte collezionabili composto da due giocatori che si sfidano in un duello in cui devono cercare di ridurre i punti vita dell'avversario a zero. Per fare ciò, i duellanti sono dotati di un
    mazzo di carte composto da mostri, magie e trappole. I mostri rappresentano delle truppe schierabili dalla parte del giocatore che possono attaccare l'avversario o difendere il
    giocatore stesso, mentre le magie e le trappole sono carte che possono influenzare lo svolgimento del gioco tramite i loro effetti. Il gioco è a turni e ogni turno è diviso in fasi. Le 
    fasi contraddistinguono le azioni che il giocatore può intraprendere durante quel periodo di tempo.\\
    \newline
    Esiste una serie animata basata su \textit{Yu-Gi-Oh!} ed è proprio da questa che nasce l'idea del progetto. Nella serie, quando il protagonista gioca una carta mostro, 
    viene proiettato un ologramma del mostro stesso, in modo che si materializzi sul campo di battaglia. L'ologramma è visibile a tutti i giocatori e può essere interagito da chiunque.\\
    Venuti a conoscenza dell'esistenza di un HMD come HoloLens, si è pensato di realizzare uno \textit{strategy-game} in realtà aumentata e \textit{real-time} che permettesse di riprodurre 
    quanto più fedelmente l'esperienza riportata nella serie.\\
    \newline
    Il progetto consiste quindi nella realizzazione di un video game distribuito, ovvero un gioco che permetta a più utenti di interagire tra loro in un ambiente virtuale condiviso. 
    Si è posto sin da subito l'obiettivo di realizzare il gioco tramite una \textit{web app}, ovvero un'applicazione che possa essere eseguita da un browser. In questo modo non ci 
    sarebbero stati problemi di portabilità, rendendolo accessibile a chiunque. In aggiunta, lo sviluppo web è ormai dominante nel settore informatico e permette di raggiungere un pubblico
    più vasto oltre che comunità più grandi e attive.\\
    La tecnologia più utilizzata in ambito web per la realizzazione di applicazioni AR è WebXR. WebXR permette di creare realtà aumentate eseguibili su browser in cui ogni utente
    vive la stessa esperienza ma in maniera indipendente. Bisognava quindi ancora trovare uno strumento per sincronizzare i vari utenti in modo da condividere l'esperienza. A supporto 
    di ciò si è utilizzato Croquet, framework che sopperisce esattamente al requisito richiesto. Si noti che tutta la sezione riguardante il supporto all'esperienza distribuita è un tema
    non trattato nel corso di studi e che è stato affrontato autonomamente. Questo ha portato però a dei tagli sulla realizzazione delle funzionalità di gioco, che sono state ridotte
    al minimo indispensabile per poter concentrare gli sforzi sullo sviluppo dell'architettura generale del sistema.\\
    \newline
    \paragraph{Al momento dell'avvio} l'utente dovrà affrontare un duello contro un'altra persona a \textit{Yu-Gi-Oh!}. Per la decisione del regolamento da seguire si è optato per
    una versione semplificata del gioco. Il giocatore potrà pescare dal mazzo una carta per turno. Nella fase appropriata, il player potrà giocare carte mostro che saranno in grado di 
    attaccare l'avversario, riducendone i punti vita, e difendere il proprio controllore dagli attacchi nemici. Saranno presenti anche carte magia e trappola con gli effetti più disparati.
    L'obiettivo del gioco consiste quindi nell'azzerare i punti vita dell'avversario, che comporterà anche la conclusione della simulazione.

    \subsection{Requisiti funzionali}\label{subsec:requisitiFunzionali}
        \begin{itemize}
            \item Il giocatore sarà in grado di vedere gli ologrammi personali e condivisi in tempo reale.
            \item Sarà visualizzato un ologramma personale che mostra i punti vita rimanenti nonchè la fase e il turno attuali.
            \item Il giocatore potrà interagire con un mazzo di carte virtuale pescando la prima carta.
            \item Il player potrà posizionare le carte che ha in mano sul campo e di conseguenza far apparire la corrispondente carta nello spazio di gioco condiviso.
            \item Il giocatore potrà avanzare di fase, come ordinare l'attacco di un mostro tramite un menù apposito.
        \end{itemize}
    \subsection{Requisiti non funzionali}\label{subsec:requisitiNonFunzionali}
        \begin{itemize}
            \item L'applicazione dovrà essere una \textit{web app} per rendere fruibile, tramite un qualsiasi browser compatibile, l'esperienza di gioco.
            \item L'applicazione dovrà essere in grado di gestire momentanee disconnessioni da parte di tutti gli utenti.
            \item L'applicazione dovrà prevedere l'ingresso anche di più di due giocatori alla sessione, mostrando ai giocatori inattivi il duello in atto.
        \end{itemize}

    \subsection{Modello del dominio}
        Il videogioco \textit{Yu-Gi-Oh!} dovrà modellare i concetti di giocatore, carta, campo di gioco, turno, mazzo e mano. Ognuno di questi concetti rappresenta un tassello fondamentale
        della struttura del video game. Il giocatore corrisponde all'utente che sta giocando, verrà fornito di un \textit{mazzo} di carte impilate e coperte e di una \textit{mano} di
        carte visibili che potrà giocare sul campo di gioco. Il campo di gioco è composto da due aree speculari, una per ogni giocatore, in cui verranno posizionate le carte giocate
        dai duellanti. Il turno rappresenta l'unità di tempo in cui un giocatore può eseguire delle azioni, come pescare una carta dal mazzo, giocare una carta dalla mano o attaccare
        con un mostro. Si noti che ognuna di queste azioni sarà eseguibile in una \textit{fase} precisa del turno, si potrà pescare una carta dal mazzo nella \textit{draw phase} ma
        non si potranno posizionare carte nel campo, si potrà attaccare con un mostro nella \textit{battle phase} ma non si potrà pescare e così via.\\
        Uno diagramma delle classi riassuntivo della trattazione appena fatta è riportato in figura~\ref{fig:ModelloDominio.png}.\\
        \img[1]{ModelloDominio.png}{Schema UML del modello del dominio.}
    