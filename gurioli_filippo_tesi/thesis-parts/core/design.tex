\section{Design architetturale}\label{sec:design}
In questo paragrafo si scenderà nel dettaglio dell'architettura del sistema, descrivendo le scelte progettuali effettuate per il corretto funzionamento delle tecnologie adottate.
\newline
Il framework di base da cui si è sviluppato il progetto è Croquet. La documentazione cita che per una buona realizzazione di un progetto Croquet bisogna cercare di mantenere una 
struttura speculare tra model e view. Nel primo vengono immagazzinate le informazioni e nella seconda vengono visualizzati i dati in base al tipo di utente che ci vuole accedere.
Ogni \texttt{Croquet.Model} dovrà essere completamente indipendente dal framework di visualizzazione, in modo da poter essere utilizzato in qualsiasi contesto. Al contrario, nelle
\texttt{Croquet.View} si dovranno trovare solo componenti visibili all'utente, senza alcuna logica di business.\\
Croquet gestisce i dati condivisi creando una loro istanza all'interno di ogni client e sincronizzandoli tramite l'utilizzo dei reflector. Per comprendere meglio il funzionamento
dei passaggi successivi, si consiglia di immaginare questi dati come all'interno di un server fittizio a cui ogni client ha accesso come mostrato in figura~\ref{fig:CroquetServer.png}.\\
\img{CroquetServer.png}{In alto la vera struttura di Croquet, in basso una rappresentazione semplificata.}
Al contrario del model, ogni view va vista come una istanza di realtà aumentata unica per ogni utente. Ogni giocatore è immerso nella propria simulazione BabylonJS con un proprio engine,
una priopria camera e i propri oggetti di scena. Queste simulazioni interagiscono tra loro grazie agli eventi: all'interazione dell'utente con gli oggetti di scena, la view invia un
evento al model che lo elabora, modifica il proprio stato e aggiorna le view degli altri utenti come mostrato in figura~\ref{fig:MultiUserBABYLON.png}.\\
\img{MultiUserBABYLON.png}{Ogni utente ha la propria istanza di BabylonJS.}
Si è strutturata una rete di \texttt{Croquet.Model} che distribuisse ogni funzionalità del dominio ad un modello diverso, in modo da tenere separati i compiti e mantenere una
gerarchia ordinata e intuitiva.\\



Cose da dire:
\begin{itemize}
\item Lungo tutta la \textit{codebase} si è deciso di non utilizzare ereditarietà, preferendo la composizione in quanto ritenuta più flessibile e meno vincolante. La struttura si può
raffigurare comunque in una relazione padre-figlio in cui però non è il figlio a estendere il padre, bensì il padre a creare e contenere il figlio.\\

\item Definiti questi princìpi cardine si sono quindi costruite le fondamenta su cui basare il video game. La classe più importante, da cui poi si sviluppa tutta la struttura, è 
\texttt{GameModel}. Questa classe di fatto non gestisce una struttura dati ma fa da contenitore per tutti i modelli del gioco. Qui si trovano i riferimenti ai modelli dei giocatori,
del turno e del campo di battaglia.\\
Altra funzionalità importante per questa classe è di gestire le connessioni e i ruoli. All'avvio, in base al numero di partecipanti già presenti, questa classe avrà il compito di 
assegnare un ruolo al nuovo utente connesso scegliendo tra \textit{player 1}, \textit{player 2} e \textit{spettatore}. Inoltre, se un utente dovesse disconnettersi, questa classe
dovrà gestire una sua possibile riconnessione come anche prevedere una sequenza di terminazione nel caso in cui l'utente non dovesse riconnettersi.\\
Nella controparte \texttt{GameView} si possono trovare gli stessi riferimenti alle view corrispondenti dei modelli citati. Si noti che, dato che il model non contiene una struttura
dati, la \texttt{GameView} non crea alcun componente visibile all'utente, mantenendo coerenza con il principio di specularità tra model e view.\\

\item Il back-end nel progetto è rappresentato da un insieme di \texttt{Croquet.Model} che si occupano di gestire i dati e di fornire le funzionalità base per interagirci. Spesso vengono
anche utilizzate classi standard JavaScript contenute all'interno dei model che fungano da supporto per gestire le strutture dati. Si è deciso di non utilizzare ereditarietà 
all'interno del progetto, preferendo la composizione, in quanto si è ritenuto che non fosse necessario avere una gerarchia di classi. 
La prima classe model creata, nonchè quella passata 
\end{itemize}