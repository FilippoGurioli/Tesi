\documentclass[12pt,a4paper,twoside,openright]{book}

\usepackage[italian]{babel}
\usepackage{style/isi-lt}

\input{style/isi-lst}

\newcommand{\code}[2]{
    \lstinputlisting[
        caption={#2}, 
        label=lst:#1
    ]{code/#1}
}

\newcommand{\img}[3][0.8] {
    \begin{figure}
        \centering
        \includegraphics[width=#1\textwidth]{images/#2}
        \caption{#3}
        \label{fig:#2}
    \end{figure}
}

%--------------------------------------------------------------------
%---------------------- INFORMAZIONI SULLA TESI ---------------------
%--------------------------------------------------------------------

\universita{Alma Mater Studiorum -- Università di Bologna}
\campus{Campus di Cesena}
\dipartimento{Dipartimento di Informatica -- Scienza e Ingegneria}
\cdl{Corso di Laurea in Ingegneria e Scienze Informatiche}

\titolo{Come si progetta un gioco in realtà aumentata multiplayer usando webxr}
\materia{Sistemi embedded ed internet of things}

\laureando{Filippo Gurioli}

\relatore[Prof.]{Alessandro Ricci}
\correlatorea[Dott. Ing.]{Samuele Burattini}

\annoaccademico{2022 -- 2023}

\dedica{eventuale dedica o citazione...}

\makeindex

\begin{document}
\frontmatter 

\maketitle
\tableofcontents

\chapter{Introduzione}
\markboth{INTRODUZIONE}{INTRODUZIONE}

Qui il testo dell'introduzione alla tesi. Generalmente l'introduzione non dovrebbe superare le 2/3 pagine e dovrebbe essere scritta solo alla fine.
	
\mainmatter

\pagestyle{fancy} 
\fancyhead[LE,RO]{\thepage}
\fancyfoot{}

\chapter{Stato dell'arte}\label{chap:Letteratura}
Di seguito si riporteranno informazioni circa lo stato dell'arte riguardante la Extended Reality, in particolare, si toccherà il tema dello sviluppo di videogiochi in ambito 
Augmented Reality e Virtual Reality. Successivamente si analizzerà la letteratura riguardante il tema del multiplayer e come questo sia attualmente realizzato nei software comuni,
sottolineando l'importanza del web al giorno d'oggi.

\section{Extended Reality}\label{sec:XR}
    La Extended Reality è un campo molto vasto che comprende diverse tecnologie ed è in continua evoluzione. Talvolta non è possibile fornire una definizione univoca dei vari 
    concetti per via delle diverse visioni che gli esperti in materia hanno. Lo scopo della sezione seguente è quindi quello di chiarire i concetti fondamentali riguardanti
    questa tecnologia, esplorare le diverse accezioni e presentare alcuni esempi di utilizzo.
    \subsection{Definizione}\label{subsec:XRDef}
        Con il termine Extended Reality (abbreviato in XR) si intende un insieme di tecnologie che permettono di estendere la realtà, ovvero di aggiungere informazioni al mondo 
        reale. Sotto questa macro definizione rientrano le tecnologie di realtà aumentata (AR), realtà virtuale (VR) e realtà mista (MR). Per AR si intende la tecnologia che permette 
        di sovrapporre informazioni digitali al mondo reale, mentre per VR si intende la tecnologia che permette di immergere l'utente in un ambiente virtuale. La MR è una tecnologia
        che permette di conciliare il mondo reale con quello virtuale in modo totalmente trasparente per l'utente. Come ogni calcolatore, queste tecnologie prevedono dispositivi 
        di input e di output per l'interazione tra macchina e utente. Venendo completamente immersi in queste realtà virtuali, l'utilizzo di una tastiera o un mouse risulterebbe
        inefficace, si sono quindi progettati dispositivi ad hoc per l'interazione (input) e per la ricezione delle informazioni (output). Per quanto riguarda l'input, si possono
        utilizzare dei controller, dei guanti o il proprio corpo, mentre per l'output domina l'utilizzo di visori. Questi dispositivi sono in grado di rilevare i movimenti 
        dell'utente e di trasmetterli al sistema, che li elabora e li utilizza per modificare l'ambiente virtuale. Questo ambiente virtuale può essere visualizzato dall'utente 
        tramite un visore, che può essere un visore VR, un visore AR o un visore MR.
        \paragraph{Augmented Reality e Mixed Reality} sono due concetti molto simili, dei quali si trovano definizioni ambigue, talvolta discordanti. La definizione più comunemente
        accettata per Augmented Reality è quella fornita da Azuma\cite{Azuma1997}:
            \begin{quote}
                \textit{AR systems have the following three characteristics: (1) combine real and virtual, (2) are interactive in real time, and (3) are registered in 3D.}
            \end{quote}
            Si noti che non è specificato nessun tipo di dispositivo di output, potrebbero essere  AR anche dispositivi che forniscono informazioni tattili, gustative o olfattive,
            benchè la loro implementazione è di fatto un settore tutt'ora nascente (motivo per il quale nella sezione precedente non li si è menzionati). La definizione formale di 
            Mixed Reality è invece fornita da Milgram\cite{Milgram1994}:
            \begin{quote}
                \textit{A mixed reality (MR) system is one that combines real and virtual environments seamlessly.}
            \end{quote}
            Questa definizione è molto simile a quella di Azuma, ma non è specificato il fatto che l'ambiente virtuale debba essere tridimensionale. La definizione di Milgram non
            specifica, inoltre, che l'ambiente virtuale debba essere sovrapposto a quello reale, ma che i due ambienti debbano essere combinati in modo da risultare indistinguibili. 
            In quest'ottica si può quindi affermare che la AR è un caso particolare di MR, in cui l'ambiente virtuale è sovrapposto a quello reale.\\
            Il libro di Schmalstieg e Höllerer \textit{Augmented Reality: Principles and Practice}\cite{Schmalstieg2016} fornisce una schematizzazione della gerarchia tra AR e MR
            mostrata in figura\ref{fig:ARvsMR} che permette di comprendere meglio la relazione tra le due tecnologie. 
            \img{ARvsMR}{Gerarchia tra AR e MR - \textit{`The mixed reality continuum'}.}
            
    \subsection{Storiografia}\label{subsec:XRstoriografia}
        Come già accennato prima, il mondo della Extended Reality non si ferma al solo utilizzo di visori. Nella storia esistono esempi di applicazioni di AR e VR molto particolari 
        e interessanti. Si presenteranno di seguito alcuni di questi esempi, cercando di esplorare le scelte architetturali più interessanti.
        \paragraph{La spada di Damocle} (figura\ref{fig:SpadaDiDamocle}) è considerato  il primo prototipo di visore VR mai costruito nella storia e, come tale, ha diritto di essere
            citato come primo elemento di questa sezione. Il visore è stato costruito da Ivan Sutherland nel 1968 ed era pesante al punto da dover essere fissato al soffito per
            essere sorretto, da questo il suo nome. Il visore era in grado di mostrare all'utente un cubo tridimensionale, che poteva essere osservato da diverse angolazioni 
            semplicemente muovendo la testa, era infatti già presente, seppur in forma molto rudimentale, il concetto di \textit{head tracking}. Il visore era inoltre dotato di un 
            controller, che permetteva di interagire con l'ambiente virtuale. Questo prototipo è stato il primo passo verso la realizzazione di visori VR, ma non è stato
            mai commercializzato.
            \img[0.5]{SpadaDiDamocle}{La spada di Damocle, primo prototipo di visore VR.}

        \paragraph{Sensorama} (figura\ref{fig:Sensorama}) è un dispositivo che permette di immergere l'utente in un
            ambiente virtuale, mostrando un video stereoscopico a 3 dimensioni e riproducendo suoni e odori. Il dispositivo è stato
            progettato per essere utilizzato in sale cinematografiche, in modo da coinvolgere lo spettatore con tutti e 5 i sensi.
            La macchina di Morton Heilig era però troppo costosa, e questo, unito alle dimensioni della macchina e alla scarsa qualità
            video, ha portato al fallimento del progetto.
            \img[0.5]{Sensorama}{Sensorama, dispositivo fornito di: immagini stereo 3D, vibrazioni, vento, emettitore di odori e audio stereo.\cite{wikiSensorama}}

        \paragraph{IllumiRoom} è una tecnologia Microsoft che punta a creare un'esperienza AR estendendo la realtà di gioco
            normale, composta dallo schermo e dal controller, all'intera stanza. Questa tecnologia è stata presentata nel 2013
            e permette di proiettare sulle pareti della stanza in cui si gioca le immagini del gioco, in modo da estendere
            l'ambiente virtuale a quello reale. Questa tecnologia è stata sviluppata per la console Xbox One, ma non è mai
            stata commercializzata.\cite{Schmalstieg2016}
            \img{IllumiRoom.png}{IllumiRoom, prodotto capace di estendere l'ambiente di gioco all'intera stanza.}
        
        \paragraph{L'HTC Vive} è uno dei più famosi visori VR, è stato sviluppato da HTC e Valve Corporation ed è
            rilasciato nel 2016. L' \textit{Head Mounted Display} (HDM) è dotato di due schermi OLED da $1080\times1200$ pixel, 
            con un refresh rate di 90Hz\cite{Vive}. Questo visore è uno dei più conosciuti sul mercato, 
            e tra le altre cose, deve anche la sua fama alla tecnologia di tracciamento dei movimenti chiamata \textit{Lighthouse}, che permette di tracciare
            i movimenti dell'utente in modo preciso e affidabile. Questa tecnologia è stata sviluppata dalla Valve
            e consiste di due stazioni base che emettono dei segnali infrarossi rilevati da sensori posti 
            sulle periferiche. Queste stazioni base vengono poste alle estremità della stanza, in modo tale da circoscrivere
            un'area entro la quale l'utente è libero di muoversi.
            \img{HTCVive.jpg}{HTC Vive, da sinistra a destra: stazioni base, controller e visore.}

        \paragraph{I dispositivi mobili} sono alla base della maggior parte delle applicazioni di AR. Questi dispositivi
            sono dotati di fotocamera, accelerometro, giroscopio, magnetometro e geolocalizzazione che permettono di rilevare i movimenti
            dell'utente e di orientare la fotocamera. Questi dispositivi sono inoltre dotati di schermo, che permette di
            visualizzare l'ambiente virtuale. I cellulari sono molto utilizzati per applicazioni di Augmented Reality, in quanto
            sono dispositivi che tutti possiedono e che sono in grado di fornire un'esperienza soddisfacente.
            Attualmente sul mercato è esploso il fenomeno degli \textit{AR advertising}, ovvero applicazioni che permettono
            di visualizzare i prodotti venduti dalle aziende in modo virtuale direttamente nello spazio reale. Queste applicazioni
            sono molto utilizzate per la vendita di mobili, in quanto permettono di proiettare il prodotto in modo realistico
            e di valutare se questo si adatta all'ambiente in cui si vuole inserire. Anche il settore videoludico non ha 
            perso l'occasione di sfruttare questa tecnologia, infatti sono state sviluppate applicazioni che permettono di
            visualizzare i personaggi dei propri giochi preferiti direttamente nella propria stanza. Un esempio di questo
            tipo di applicazioni è \textit{Pokémon GO}, gioco sviluppato da Niantic nel 2016, che permette di catturare i
            Pokémon direttamente nella propria città.
            %\img{PokemonGo.png}{Pokémon GO, applicazione di AR per dispositivi mobili.}
            
        \paragraph{Hololens} è il dispositivo Microsoft rilasciato nel 2016 che ha portato il mondo delle tecnologie 
            AR e MR a un livello superiore. Questo HMD è dotato di un processore di fascia alta, una GPU per il calcolo
            di immagini ed una Holographic Processing Unit (HPU) che permette di elaborare i dati provenienti dai sensori
            in tempo reale. Il visore monta anche una telecamera di profondità, sensori avanzati per il tracciamento dei
            movimenti e un sistema di rilevamento del suono. Questo visore è in grado di proiettare immagini in modo
            realistico sulle lenti con display ottico 3D, in modo da sembrare che l'immagine sia davanti all'utente.
            Il visore ha un costo ancora troppo elevato per essere utilizzato da un utente comune, ma è stato utilizzato
            in ambito industriale per la progettazione di automobili e per la formazione di chirurghi. \\
            Tra le varie feature che il visore offre ci sono:
            \begin{itemize}
                \item \textbf{Spatial Mapping}: permette di mappare l'ambiente circostante e di rilevare gli oggetti
                    presenti, in modo da poterli utilizzare per interagire con l'ambiente virtuale.
                \item \textbf{Spatial Sound}: permette di rilevare la posizione dell'utente e di modificare il suono in base
                    ad essa.
                \item \textbf{Gaze Tracking}: permette di rilevare la direzione dello sguardo dell'utente.
                \item \textbf{Voice Recognition}: permette di rilevare i comandi vocali dell'utente.
                \item \textbf{Gesture Recognition}: permette di rilevare i movimenti delle mani dell'utente svincolando
                    l'utente dall'uso di controller e guanti (un ulteriore passo avanti verso la \textit{computazione ubiqua}
                    \footnote{Computazione ubiqua: modello post-desktop di interazione uomo-macchina, in cui l'elaborazione
                    delle informazioni è stata interamente integrata all'interno di oggetti e attività di tutti i giorni.\cite{ubiCompWiki}}).
            \end{itemize}
            In conclusione si noti che ogni sensore è completamente integrato nel visore, che non necessita quindi di
            dispositivi esterni rendendolo completamente wireless.
            \img{Hololens.png}{Hololens, visore AR/MR di Microsoft.}

    \subsection{Applicazioni}\label{subsec:XRapplicazioni}
        Le applicazioni di Extended Reality sono molteplici e vanno dai videogiochi, all'industria fino alla medicina. In questa sezione si analizzeranno alcune applicazioni di AR e VR, 
        cercando di capire quali sono le scelte architetturali più interessanti.
        \paragraph{L'industria} sta utilizzando queste nuove tecnologie per migliorare la produttività e la qualità dei prodotti. Un esempio di questo utilizzo è quello già
            accennato di AR Ads per la vendita di mobili, vestiti e altre merci, di cui i principali esempi sono:
            \begin{itemize}
                \item American Apparel che ha sviluppato un'applicazione in grado di mostrare i vestiti in modo realistico sul proprio corpo;
                \item IKEA che ha sviluppato un'applicazione che permette di visualizzare i propri mobili direttamente nella propria casa;
                \item De Beers che ha sviluppato un'applicazione che permette di provare gli anelli direttamente sul proprio dito.
            \end{itemize} 
            Un altro esempio di utilizzo di AR in ambito industriale è quello di \textit{Remote Assistance}, ovvero l'assistenza remota. Questa tecnologia permette di fornire 
            assistenza a distanza, dando la possibilità di vedere ologrammi illustrativi direttamente nello spazio reale, facilitando il lavoro dei tecnici e permettendo di 
            ridurre i costi di formazione e di spostamento.\\

        \paragraph{Augmented Medicine} è il termine che è stato coniato per indicare l'utilizzo di dispositivi AR in ambito medico. Tra queste applicazioni si possono citare:
            \begin{itemize}
                \item \textbf{Formazione di chirurghi}: in quanto permette di visualizzare in modo realistico gli organi 
                    interni del paziente, permettendo di eseguire operazioni in modo virtuale e di simulare interventi chirurgici.
                \item \textbf{Progettazione di protesi}: in quanto permette di visualizzare in modo realistico le protesi 
                    direttamente sul paziente, permettendo di valutare se la protesi si adatta al corpo del paziente.
                \item \textbf{Vista in profondità}: l'AR ha permesso a chirurghi di vedere in tempo reale tumori, fratture e aneurismi direttamente sul paziente, in modo da poter 
                    agire in modo più preciso e sicuro.
            \end{itemize}
            \img{ARMedicine.png}{Esempio di applicazione di AR in ambito medico.}

        \paragraph{I videogiochi} rappresentano il settore che più di tutti ha sfruttato le potenzialità di queste tecnologie. I videogiochi sono stati sviluppati per essere giocati 
            su schermi bidimensionali, ma con l'avvento di queste nuove tecnologie si è aperto un nuovo mondo di possibilità. Sin dagli inizi della realtà virtuale si possono notare
            i primi prodotti di successo come \textit{Virtuality}, \textit{BeatSaber} e \textit{Half-Life: Alyx}. Queste applicazioni hanno sfruttato le potenzialità di queste
            tecnologie per creare esperienze di gioco uniche, coinvolgenti e realistiche.\\
            Per avere una panoramica più dettagliata di queste applicazioni si rimanda alla sezione successiva (\ref{sec:Videogiochi}) dove verranno spiegati in dettaglio
            il significato, la storia e le categorie più famose dei video games.

\section{Videogiochi}\label{sec:Videogiochi}
    In questa sezione si parlerà di videogiochi, in particolare si darà una definizione formale, si analizzerà la storia di questo settore discutendone le applicazioni più famose e 
    si parlerà delle varie tipologie di videogiochi.
    
    \subsection{Definizione}\label{subsec:VideogiochiDef}
        Da quando il settore videoludico ha fatto la sua ascesa il termine `gioco' non ha più avuto lo stesso significato. La parola `gioco' viene utilizzata per indicare un'attività 
        ludica, che ha come scopo il divertimento, la spensieratezza e il passatempo. Tra le varie attività ludiche si possono annoverare i giochi da tavolo (Monopoli, Risiko, etc.), i 
        giochi di carte (scala quaranta, bridge, etc.), i giochi sportivi (frisbee, calcio, etc.) o i semplici giochi tradizionali per bambini (strega comanda colore, uno due tre stella, etc.).
        È in questo contesto che il videogioco prende piede, mantenendo l'accezione comune di gioco come attività ludica con un regolamento e un'ambientazione, ma aggiungendo un elemento
        che lo distingue da tutti gli altri giochi: l'interazione con un dispositivo elettronico. \\
        Il termine videogioco significa `gioco gestito da un dispositivo elettronico'. Sono videogiochi quindi i conosciutissimi giochi online, i giochi per cellulare, i giochi per console 
        e i giochi per computer, ma anche i cabinati e le slot machine soddisfano questa definizione in quanto non solo hanno un software che gestisce il gioco, ma anche un hardware dedicato.\\
        Il termine `videogioco' è stato coniato da Ralph Baer nel 1966, utilizzato dall'ingegnere per descrivere il proprio progetto di gioco elettronico (di fatto la prima console della storia).
        Ai giorni d'oggi questo termine ha un significato quasi completamente distaccato dall'accezione generale di gioco, esistono comunità intere che fanno dei video games una professione,
        esistono competizioni internazionali e campionati mondiali e ci sono anche videogiochi che hanno un budget di produzione maggiore di quello di un film. In questo contesto anche l'utente
        medio si approccia al gioco in modo più serio, non più come un semplice passatempo, ma come un'esperienza da vivere e da condividere con gli altri. Quello che si può definire come
        un vero e proprio fenomeno culturale ha stravolto molti paradigmi: il tempo che si dedica a conoscere il gioco, sia direttamente giocandolo, che indiriettamente informandosi su di esso,
        l'accezione che ormai ha perso il sinonimo di passatempo, il luogo in cui si gioca ed il modo in cui si gioca.\\ 

    \subsection{Storiografia}\label{subsec:VideogiochiStoriografia}
        La storia dei videogiochi è molto lunga e complessa, in questa sezione si cercherà di riassumere i punti salienti di questa storia passando per gli esempi più famosi e le
        innovazioni più importanti.

        \paragraph{Pong} è il videogioco per antonomasia, precursore della storia dei video games. Questo gioco è stato sviluppato da Atari nel 1972 ed è stato il primo gioco arcade
            di successo. Il gioco consiste in una simulazione di ping pong in cui due giocatori si sfidano a colpi di palla. In questo semplice esempio si concretizza già il primo 
            multigiocatore, Pong è infatti giocabile sia in singolo, contro il computer, che in due giocatori, mostrando sin da subito che un gioco ha più valore se giocato in compagnia. 
            Bisogna ricordare Pong anche come uno dei primi \textit{coin-op} della storia, ovvero uno dei primi giochi arcade che richiedevano l'inserimento di una moneta per poter 
            essere giocati.
            \img{Pong2.png}{Pong, il primo videogioco arcade di successo.}

        \paragraph{Pac-Man} è il gioco ideato da Toru Iwatani, prodotto dalla Namco, che ha appassionato molti nelle salegiochi degli anni '80. Il gioco consiste di un personaggio giallo,
            comandato dal giocatore, che ha lo scopo di mangiare tutti i puntini presenti nel labirinto, evitando di essere mangiato dai fantasmi. Esistono anche delle `pillole speciali'
            ai lati del labirinto che permettono al giocatore di diventare immune ai fantasmi per 10 secondi. Infine, se il giocatore riesce a mangiare tutti i puntini del labirinto,
            viene mostrato un intermezzo, per poi far ricominciare il gioco da capo (mantenendo il punteggio). Il gioco deve il suo successo a diversi fattori tra cui: 
            essere uno dei pochi giochi non violenti in un mercato dominato da giochi di guerra, avere dei comandi semplici e intuitivi e avere un gameplay basato più sulla tattica 
            che sui riflessi\cite{Uston1982}.
            %\img{PacMan.png}{Pac-Man, gioco arcade in cui controlli pacman (personaggio giallo) e sei inseguito dai fantasmi (colorati in diversi colori).}

        \paragraph{Super Mario Bros.} è la punta di diamante della Nintendo che pur non essendo il primo Mario creato, rimane comunque il più celebre. Questo gioco ha gettato le fondamenta
            per tutti i giochi platform che si sarebbero creati negli anni successivi. Questo consiste in un platform a scorrimento orizzontale in cui il giocatore controlla Mario, 
            un idraulico che deve salvare la principessa Peach dal malvagio Bowser. È presente anche la possibilità di giocare in cooperativa con un altro giocatore, che controllerà
            Luigi, il fratello di Mario. Il video game di Miyamoto ha portato un vento di innovazione nel settore, portando a fama mondiale il concetto livelli non lineari, ovvero
            livelli in cui il giocatore può scegliere il percorso da seguire, e il concetto di power-up, ovvero oggetti che permettono al giocatore di ottenere abilità speciali.\cite{ryan2011}
            %\img{SuperMarioBros.jpg}{Super Mario Bros., platform che ha rivoluzionato il settore videoludico.}
            
        \paragraph{Doom} è stato il primo \textit{First Person Shooter} (FPS) mai creato. Il gioco, in sè semplice e consiste di un marine, controllato dal giocatore, che deve sopravvivere
            all'invasione di demoni provenienti dall'inferno. Questo gioco è stato sviluppato da id Software nel 1993 ed è stato il primo gioco a utilizzare la tecnologia \textit{ray casting} 
            per la generazione di ambienti tridimensionali. Per la prima volta si riuscì a creare un ambiente tridimensionale partendo da un piano bidimensionale. La tridimensionalità era
            però solo apparente, la vera struttura del livello era bidimensionale, non permettendo di sviluppare ambienti sul terzo asse. Queste limitazioni non hanno
            però impedito al videogioco di avere successo a livello globale, portando alla creazione di un genere di giochi che ancora oggi è molto popolare.\\
            Altra caratteristica innovativa di Doom è la possibilità di giocare in multiplayer online, permettendo a macchine diverse di collegarsi alla stessa sessione di gioco
            se connesse alla stessa LAN, lasciando agli utenti anche la scelta tra un gioco cooperativo (\textit{coop}) o competitivo (\textit{deathmatch}).
            \img{Doom.png}{Doom, primo FPS della storia.}
        
        \paragraph{Minecraft} è il gioco attualmente più venduto al mondo, con oltre 238 milioni di copie vendute. Il famosissimo \textit{sandbox} sviluppato dalla Mojang Studios è stato
            rilasciato per la prima volta nel 2011 e da allora ha avuto un successo inarrestabile. Consiste in un mondo tridimensionale generato proceduralmente, in cui il giocatore
            può costruire e distruggere blocchi di vario tipo. Il gioco è stato sviluppato interamente in Java permettendo così agli utenti di creare mod e plugin per modificarlo
            a proprio piacimento. Questo video game è supportato su tutte le piattaforme principali: da Windows a Linux, dalle console al cellulare rendendolo accessibile ad un grande
            numero di utenti. Minecraft ha avuto anche un grande impatto sul gioco online, portando alla creazione di server dedicati (i \textit{Realms}), in cui gli utenti possono 
            giocare insieme, rendendo disponibile anche la possibilità del cross-platform all'interno di questi server.
            %\img{Minecraft.png}{Minecraft, sandbox multiplayer campione di incassi.}

    \subsection{Categorie}\label{subsec:VideogiochiCategorie}
        I videogiochi sono stati suddivisi in categorie in base a diversi criteri, tra cui: la prospettiva, il genere, il gameplay, il target di riferimento e la piattaforma.
        Ogni videogioco può essere classificato in base a uno o più di questi criteri. In questa sezione si cerca di identificare le categorie principali, specialmente quelle riguardanti
        l'elaborato di tesi. Le categorie così identificate sono: avventura, azione, di ruolo, strategico, online.
        \begin{itemize}
            \item \textbf{Avventura} - Questa categoria comprende tutti quei giochi in cui il giocatore deve esplorare un mondo virtuale, risolvere enigmi e interagire con gli NPC 
                (personaggi non giocanti). Questi giochi sono caratterizzati da una trama che si sviluppa nel corso del gioco, in cui il giocatore deve raggiungere degli obiettivi per 
                poter procedere. I sottogeneri principali sono: avventura testuale (di cui il gioco più famoso è \textit{Zork}), avventura grafica (di cui il gioco più famoso è
                \textit{Monkey Island}) e avventura dinamica (di cui il gioco più famoso è \textit{The Legend of Zelda}).
            \item \textbf{Azione} - Questo genere comprende videogiochi in cui riflessi e agilità nella combinazione di comandi sono determinanti per la vittoria. Questi giochi sono 
                infatti caratterizzati da un gameplay frenetico unito ad una difficoltà crescente. I sottogeneri principali sono: platform (con \textit{Super Mario Bros.} 
                come gioco rappresentativo), picchiaduro (con \textit{Street Fighter}), sparatutto (con \textit{Doom}) e battle royale (con \textit{Fortnite}).
            \item \textbf{Di ruolo (RPG)} - Questo genere è caratterizzato da un gameplay in cui il giocatore controlla un personaggio che può essere personalizzato e che può 
                migliorare le proprie abilità nel corso del gioco. Fondante è anche la componente narrativa, i ruoli dei vari personaggi e le classi a cui appartengono. I sottogeneri
                principali sono: \textit{action RPG} (con \textit{Dark Souls} come capostipite), \textit{MMORPG}\footnote{MMORPG: massively multiplayer online RPG, RPG online che ammettono
                all'interno della stessa sessione un elevato numero di partecipanti, nell'ordine delle centinaia.} (con \textit{World of Warcraft}) e \textit{roguelike} (con \textit{The
                Binding of Isaac}).
            \item \textbf{Strategico} - Questa categoria comprende tutti quei giochi in cui le scelte che il giocatore prende hanno un impatto determinante sullo svolgimento del gioco.
                I sottogeneri principali sono: strategia in tempo reale (con \textit{Age of Empires} come gioco rappresentativo), strategia a turni (con \textit{Civilization}) e 
                strategia a squadre (con \textit{League of Legends}).
            \item \textbf{Online} - Questa categoria è trasversale a quelle precedenti, in quanto comprende tutti quei giochi che permettono di giocare online con altri giocatori. 
                Questi giochi possono essere uno qualsiasi dei generi o sottogeneri menzionati prima, l'unica prerogativa è che permettano di collegare più giocatori nella stessa partita.
                Per una visione più amplia e dettagliata dei giochi online e più genericamente dei giochi multigiocatore si rimanda alla sezione successiva (\ref{sec:Multigiocatore}).
        \end{itemize}

\section{Multigiocatore}\label{sec:Multigiocatore}
    Se si vuole trovare un caposaldo all'interno di tutta la trattazione che è stata fatta precedentemente sulla definizione di gioco, questo è sicuramente la condivisione dell'esperienza. 
    Il gioco è tale solo se ci sono una o più persone con cui condividerlo, si può quasi affermare che sia intrisenco dentro la natura umana. Questa volontà di condivisione 
    non poteva quindi mancare anche nel mondo dei videogiochi, che ha espresso questa volontà nelle forme più disparate. In questa sezione si parlerà di multigiocatore, passando per 
    i punti salienti della sua storia e per le sue applicazioni più famose.
    \subsection{Definizione}\label{subsec:MultiDef}
        Il significato di multigiocatore è derivabile direttamente dalla parola stessa, è un gioco a cui può giocare più di un giocatore contemporaneamente. Questo termine è divenuto 
        di uso comune solo con l'avvento dei videogiochi ma rientrano in questa categoria anche i normali giochi da tavolo, i giochi di carte, gli sport etc.\\
        Nel contesto videoludico il termine sta a significare che nello stesso mondo virtuale, nella stessa sessione di gioco, possono partecipare più persone, le quali posso interagire
        tra loro e con l'ambiente circostante. Si noti che in questa definizione non è specificato il fatto che i giocatori debbano essere fisicamente vicini, tanto meno che ci 
        debba essere una connessione internet a collegarli. Questo è dovuto al fatto che lungo il corso della storia si sono realizzate le più disparate tecnologie per permettere la
        condivisione di gioco, alcune delle quali poi abbandonate, mentre altre valide tutt'ora. Si va quindi ad analizzare la storia del multigiocatore, con particolare attenzione 
        alle tecnologie adottate per realizzarlo.\\
    \subsection{Storiografia}\label{subsec:MultiStoriografia}
        Questa trattazione storica fa principalmente riferimento al libro \textit{Multiplayer Game Programming:  architecting networked games} di Joshua Glazer e Sanjay Madhav\cite{glazer2015}.
        \paragraph{Il Multigiocatore locale} caratterizza i primi anni in cui il concetto è stato introdotto (1960). Questa tecnologia consiste nello sviluppare un programma che sia
            capace di ricevere input da due utenti, o con la condivisione del dispositivo di input o con l'utilizzo di due dispositivi di input diversi, e, nel caso
            fosse necessario, mostrare due schermate distinte (nasce il concetto di \textit{split screen}). I videogiochi così costruiti sono quindi sviluppati sulla stessa macchina 
            e non differiscono troppo da un gioco a giocatore singolo. I primi esempi creati sono stati \textit{Pong} e \textit{Spacewar!}, entrambi giochi arcade che permettevano di 
            giocare in due giocatori.

        \paragraph{\textit{Networked multiplayer}} è la prima tecnologia che ha permesso di far comunicare due o più macchine tra loro con lo scopo di giocare su dispositivi separati. 
            Queste macchine erano in realtà ancora dei mainframe, ma si stavano gettando le basi per quello che sarebbe diventato poi internet. Nel momento in cui i personal computer
            diventeranno accessibili al pubblico, questa tecnologia verrà utilizzata per permettere a due o più giocatori di connettersi insieme, sfruttando le porte seriali dei computer. 
            Allora ogni pc aveva solo 2 porte seriali permettendo una solo topologia di rete: la \textit{daisy chain} (figura\ref{fig:DaisyChain.png}).
            \img{DaisyChain.png}{Topologia di rete \textit{daisy chain}.}
 
        \paragraph{\textit{Local Area Network Games}} sono una tipologia di giochi, venuti alla luce intorno agli anni '90, in cui i giocatori possono giocare insieme se connessi alla stessa
            rete locale (Local Area Network o LAN). Questo tipo di video game è esploso grazie all'allora nuova tecnologia, l'Ethernet. Le topologie di rete potevano variare,
            la daisy chain menzionata prima è intrinsecamente già una LAN, ma quelle più diffuse furono le connessioni a stella (figura\ref{fig:StarNetwork.png}) grazie alla grande 
            proliferazione dei modem. Il gioco più famoso di questo genere era \textit{Doom}, che permetteva di giocare in multiplayer fino a 4 giocatori.
            \img[0.5]{StarNetwork.png}{Topologia di rete a stella.}

        \paragraph{Il multigiocatore online} è forse la tecnologia più comune al giorno d'oggi ed identifica una connessione tra giocatori a livello globale (spesso si usa il termine
            Wide Area Network o WAN). Questa nuova tecnologia, seppur possa sembrare semplicemente un'estensione di quella precedente, porta con sè una serie di problematiche
            che prima non esistevano, prima tra tutte la latenza. La latenza è il tempo che intercorre tra l'invio di un messaggio in rete e la ricezione dello stesso. In una rete
            LAN questo tempo è trascurabile, ma quando ci si sposta ad un contesto globale assume importanza prioritaria (nell'ordine di qualche secondo). Questa problematica, insieme
            ad altre, porterà alla creazione di nuove infrastrutture fondate proprio per la loro gestione e risoluzione.\\
            In questo periodo diventano mutliplayer online anche le console che, grazie a servizi creati ad hoc, permettono di giocare con giocatori di tutto il mondo. 
        
        \paragraph{\textit{Massively Multiplayer Online}} o MMO, sono giochi online che permettono di giocare con un numero elevato di giocatori, nell'ordine delle centinaia. Questi
            giochi sono stati resi possibili grazie all'avvento di internet e sono stati resi popolari grazie al titolo \textit{World of Warcraft}, che ha raggiunto il picco di 12 milioni di
            giocatori attivi nel 2010 acquisendo il titolo di MMORPG più giocato al mondo. Le sfide che caratterizzano questo tipo di videogiochi sono molteplici: gestione server, 
            caching, bilanciamento del carico e tanti altri, ognuno dei quali deve essere gestito con la massima attenzione per la riuscita del videogioco.

        \paragraph{\textit{Mobile Networked Multiplayer}} è l'ultima frontiera del multigiocatore, ovvero la possibilità di connetere più dispositivi mobili alla stessa sessione di 
            gioco. Date le limitazioni di batteria (e in generale di hardware) che questi dispositivi hanno, si è dovuto trovare un modo per ridurre al minimo il consumo di dati,
            portando alla creazione di giochi turn-based asincroni, ovvero dei giochi a turni in cui ogni giocatore ha a disposizione molto tempo per fare la sua mossa e viene
            notificato quando l'avversario ha compiuto la sua.\\
            Con la crescita esponenziale di questo settore, sono diventati sempre più comuni anche i giochi in tempo reale, alcuni di questi con supporto anche di un numero cospiquo
            di utenti (nell'ordine delle decine).
        
    \subsection{Applicazioni}\label{subsec:MultiApplicazioni}
        In quest'utlima sezione del capitolo si introdurranno le architetture più comuni per la realizzazione di un videogioco multiplayer, analizzando i pro e i contro di ognuna.
        \paragraph{L'architettura client-server} consiste nella costruzione di un server, ovvero un computer che si occupa di gestire la sessione di gioco, e di più client, ovvero
            dispositivi che si connettono al server per poter giocare. Questa è la più comunemente diffusa nell'ambito videoludico, in quanto permette di gestire al meglio la
            sessione di gioco e di bilanciare il carico di lavoro tra i vari client. In questo scenario chi detiene l'informazione (ovvero lo stato del gioco) è il server, mentre
            i client hanno solo il compito di visualizzare i dati contenuti all'interno del server. Già dai primi giochi che hanno utilizzato questa architettura
            si è riscontrata la necessità di ottimizzare le informazioni che vengono passate in rete, per permettere un'esperienza piacevole e coinvolgente. Per fare ciò i 
            programmatori hanno creato varie infrastrutture che permettono di inviare solo le informazioni necessarie. Alcuni esempi di queste infrastrutture sono: 
            \textit{delta encoding}, \textit{event management} e \textit{interest management}.\\
            Il \textit{delta encoding} consiste nel trasmettere solo le informazioni che sono cambiate rispetto all'ultimo aggiornamento, questo garantisce comunque che tutti
            i client siano allineati con il server, ma permette di ridurre il carico di lavoro.\\
            L'\textit{event management} consiste nel trasformare le interazioni dell'utente in eventi. Questi eventi vengono quindi messi in una coda gestita da un 
            \textit{event manager}, che, appena possibile, invia questi eventi al server. In questo modo si possono gestire in modo efficiente le interazioni dell'utente inviando
            solo gli eventi che sono rilevanti per il giocatore. Successivamente l'evento sarà gestito dal server e il risultato sarà inviato a tutti i client.\\
            L'\textit{interest management} consiste nel trasmettere solo le informazioni che sono rilevanti per il giocatore, ovvero solo le informazioni che sono vicine al suo 
            personaggio. Questa accortezza permette anche, a chi dispone di un pc più prestante, di avere un'esperienza di gioco più immersiva e completa e, al contempo, permettere 
            a chi ha un computer dalle prestazioni basse di partecipare alla stessa sessione fornendogli comunque le informazioni necessarie per giocare. Questo meccanismo è stato
            determinante per la scalabilità di certi videogiochi come \textit{Starsiege: Tribes}.\\

        \paragraph{L'architettura peer-to-peer} o p2p è una sorta di daisy chain online. È la versione serverless in cui tutti gli utenti sono connessi tra di loro e si scambiano le 
            informazioni direttamente. Il server, avendo il duplice compito di calcolare lo stato di gioco e di gestire le connessioni, deve essere obbligatoriamente una macchina 
            dalle prestazioni molto elevate, e, conseguentemente, molto costosa. Il peer-to-peer permette di svincolare dall'acquisto di un server, garantendo anche un miglioramento 
            delle prestazioni, lasciando però il compito di mantenere uno stato di gioco coerente agli utenti.\\
            Per gestire lo stato di gioco sono stati progettati due algoritmi: \textit{lockstep} e \textit{event synchronization}.\\
            L'algoritmo \textit{lockstep} consiste nel far partire la sessione di gioco solo quando tutti i giocatori sono pronti. Questo permette di avere un'esperienza di gioco
            fluida e coerente, ma ha il difetto di non poter gestire bene i giocatori che hanno una connessione lenta, in quanto il gioco non può procedere finchè tutti i giocatori
            non hanno ricevuto le informazioni.\\
            L'algoritmo \textit{event synchronization}, al contrario, consiste nel far partire la sessione di gioco non appena un giocatore è pronto. Questo permette di gestire meglio
            i giocatori con connessioni lente, ma ha il difetto di non poter garantire un'esperienza di gioco fluida e coerente, in quanto i giocatori potrebbero ricevere informazioni 
            in ritardo rispetto agli altri giocatori.\\
            Altro aspetto fondamentale riguardante il p2p è la gestione della randomicità. Se prima nel cient-server quando bisognava generare un numero casuale si utilizzava il
            generatore di numeri casuali del server, ora, non avendo più un server, bisogna trovare un modo per generare un numero casuale che sia uguale per tutti i giocatori.
            Per fare ciò si utilizza un algoritmo che prende in input un seme e genera un numero casuale. Questo seme viene generato dal primo giocatore che si connette alla sessione
            di gioco e viene inviato a tutti gli altri giocatori. In questo modo tutti i giocatori genereranno gli stessi numeri casuali nello stesso ordine. Rimane imperativo 
            che gli accessi fatti al generatore siano gli stessi per tutti i giocatori, lasciando quindi un margine di errore non ignorabile.
    \newline \newline
    In questo capitolo si sono affrontati i temi principali che verranno trattati nel resto dell'elaborato. Arrivati a questo punto si conoscono i concetti fondamentali di 
    Extended Reality, videogiochi e multigiocatore, si conosce la loro storia e le loro applicazioni. Se in questo capitolo si è cercato di spiegare `cosa' sono questi concetti, 
    nel successivo si affronterà invece il `come' questi concetti sono stati applicati al caso specifico del progetto di tesi.
%\chapter{Progetto}\label{chap:Progetto}
Questo capitolo si pone l'obiettivo di introdurre il lettore agli argomenti trattati nei capitoli successivi, dove vengono affrontati dettagli tecnici del progetto. Qui si esporranno
in maniera generale l'idea alla base del progetto, le sue sfide principali e il traguardo che si pone di conseguire.\\
\newline
Il software che si è sviluppato consiste in una rivisitazione del gioco \textit{Yu-Gi-Oh!} trasportato in realtà aumentata. \textit{Yu-Gi-Oh!} è un gioco di 
carte collezionabili in cui due giocatori si sfidano in un duello nel quale devono cercare di ridurre i punti vita dell'avversario a zero. Per fare ciò, i duellanti sono dotati di un
mazzo di carte composto da mostri, magie e trappole. I mostri rappresentano delle truppe schierabili dalla parte del giocatore che possono attaccare l'avversario o difendere il
giocatore stesso, mentre le magie e le trappole sono carte che possono influenzare lo svolgimento del gioco tramite i loro effetti. Il gioco è a turni e ogni turno è diviso in fasi. Le 
fasi contraddistinguono le azioni che il giocatore può intraprendere durante quel periodo di tempo.\\
\newline
Esiste anche una serie animata basata su questo gioco di carte ed è proprio da questa che nasce l'idea del progetto. In questa serie, quando il protagonista gioca una carta mostro, 
viene proiettato un ologramma del mostro stesso che si materializza sul campo di battaglia. L'ologramma è visibile a tutti i giocatori e può essere interagito da chiunque.\\
Venuti a conoscenza dell'esistenza di un HMD come HoloLens, si è pensato di realizzare uno \textit{strategy-game} in realtà aumentata e \textit{real-time} che permettesse di riprodurre 
quanto più fedelmente l'esperienza riportata nella serie.
\newline
Il progetto consiste quindi nella realizzazione di un video game distribuito, ovvero un gioco che permetta a più utenti di interagire tra loro in un ambiente virtuale condiviso. 
Si è posto sin da subito l'obiettivo di realizzare il gioco tramite una \textit{web app}, ovvero un'applicazione web
che può essere eseguita da un browser. In questo modo non ci sarebbero stati problemi di portabilità, rendendolo accessibile a chiunque. In aggiunta, 
lo sviluppo web è ormai dominante nel settore informatico e permette di raggiungere un pubblico più vasto oltre che comunità più grandi e attive.\\
La tecnologia più utilizzata in ambito web per la realizzazione di applicazioni AR è WebXR. WebXR permette di creare realtà aumentate eseguibili su browser in cui ogni utente
vive la stessa esperienza ma in maniera indipendente. Bisognava quindi ancora trovare uno strumento per sincronizzare i vari utenti in modo da condividere l'esperienza. A supporto 
di ciò si è utilizzato Croquet, framework che sopperisce esattamente il requisito posto. Si noti che tutta la sezione riguardante il supporto all'esperienza distribuita è un tema
non trattato nel corso di studi e che è stato affrontato autonomamente. Questo ha portato però a dei tagli sulla realizzazione delle funzionalità di gioco, che sono state ridotte
al minimo indispensabile per poter concentrare gli sforzi sullo sviluppo dell'architettura generale del sistema.\\
\newline
Per conoscere nel dettaglio quali tecnologie si è utilizzato, come funzionano e perchè sono state scelte si rimanda al capitolo \ref{chap:Tecnologie}.
\chapter{Tecnologie}\label{chap:Tecnologie}
Elenco delle tecnologie utilizzate nell'elaborato.
\section{Hololens}\label{sec:Hololens}
\section{MRTK}\label{sec:MRTK}
\section{WebXR}\label{sec:WebXR}
\section{NodeJS}\label{sec:NodeJS}
\section{BabylonJS}\label{sec:BabylonJS}
\section{Croquet}\label{sec:Croquet}
\chapter{Progetto}\label{chap:Sviluppo}
In questo capitolo si affronteranno gli aspetti implementativi del progetto. Si inizierà con l'analisi dei requisiti, per poi passare alla fase di studio del design architetturale e
finire con la progettazione e l'implementazione vera e propria.

\section{Analisi dei requisiti}\label{sec:Analisi}
    L'obiettivo del progetto è creare un ambiente di realtà aumentata condivisa
    in cui l'utente possa giocare contro un altro al gioco di carte \textit{Yu-Gi-Oh!}. L'esperienza che il giocatore proverà dovrà essere quanto più simile alla versione
    proposta nella serie animata omonima.\\
    \newline
    Al momento dell'avvio l'utente dovrà affrontare un duello contro un'altra persona a \textit{Yu-Gi-Oh!}. Per la decisione del regolamento da seguire si è optato per
    una versione semplificata del gioco. Il giocatore potrà giocare carte mostro che
    rappresentano delle truppe schierate dalla parte del possessore. Queste truppe
    potranno quindi attaccare l'avversario per ridurne i punti vita e difendere il proprio controllore dagli attacchi avversari. Saranno presenti anche carte magia e trappola che,
    tra i vari effetti, potranno modificare i punti vita, l'ambiete di gioco in cui gli utenti giocano o anche l'attacco e
    la difesa dei mostri propri e avversari. L'obiettivo del gioco consiste quindi
    nell'azzerare i punti vita dell'avversario, che comporterà anche la conclusione della
    simulazione.

    \subsection{Requisiti funzionali}\label{subsec:requisitiFunzionali}
        \begin{itemize}
            \item Il giocatore sarà in grado di vedere gli ologrammi personali e condivisi in tempo reale;
            \item il giocatore potrà interagire con un mazzo di carte virtuale pescando la prima carta;
            \item il giocatore potrà posizionare le carte che ha in mano sul campo e di conseguenza far apparire la corrispondente carta nello spazio di gioco condiviso;
            \item il giocatore potrà avanzare di fase, come ordinare l'attacco di un mostro tramite un menù apposito;
            \item ad ogni danno (o cura) inflitto (o subito) verrà visualizzato un ologramma condiviso che mostra i punti vita rimanenti.
        \end{itemize}
    \subsection{Requisiti non funzionali}\label{subsec:requisitiNonFunzionali}
        \begin{itemize}
            \item L'applicazione dovrà sfruttare la tecnologia WebXR per rendere fruibile, tramite un qualsiasi browser compatibile, l'esperienza di gioco.
            \item L'applicazione dovrà essere in grado di gestire momentanee disconnessioni da parte di tutti gli utenti.
            \item L'applicazione dovrà prevedere l'ingresso anche di più di due giocatori alla sessione, mostrando ai giocatori inattivi il duello in atto.
        \end{itemize}

    \subsection{Modello del dominio}
        Il videogioco \textit{Yu-Gi-Oh!} dovrà modellare i concetti di giocatore, carta, campo di gioco e turno. Ognuno di questi concetti rappresenta un tassello fondamentale
        della struttura del video game. Il giocatore corrisponde all'utente che sta giocando, verrà fornito di un \textit{mazzo} di carte impilate e coperte e di una \textit{mano} di
        carte visibili che potrà giocare sul campo di gioco. Il campo di gioco è composto da due aree speculari, una per ogni giocatore, in cui verranno posizionate le carte giocate
        dai duellanti. Il turno rappresenta l'unità di tempo in cui un giocatore può eseguire delle azioni, come pescare una carta dal mazzo, giocare una carta dalla mano o attaccare
        con un mostro. Si noti che ognuna di queste azioni sarà eseguibile in una \textit{fase} precisa del turno, si potrà pescare una carta dal mazzo nella \textit{draw phase} ma
        non si potranno posizionare carte nel campo, si potrà attaccare con un mostro nella \textit{battle phase} ma non si potrà pescare e così via.\\
        \img{ModelloDominio.png}{Schema UML del modello del dominio}
    

\section{Design architetturale}\label{sec:design}
In questo paragrafo si scenderà nel dettaglio dell'architettura del sistema, descrivendo le scelte progettuali effettuate per il corretto funzionamento delle tecnologie adottate.\\
\newline
Il framework di base da cui si è sviluppato il progetto è Croquet. La documentazione cita che per una buona realizzazione di un progetto Croquet bisogna cercare di mantenere una 
struttura speculare tra model e view. Nel primo vengono immagazzinate le informazioni e nella seconda vengono visualizzati i dati. Il \texttt{Croquet.Model} dovrà essere 
completamente indipendente dal framework di visualizzazione, in modo da poter essere utilizzato in qualsiasi contesto. Al contrario, nelle
\texttt{Croquet.View} si dovranno trovare solo componenti visibili all'utente, senza alcuna logica di business.\\
Croquet gestisce i dati condivisi creando una loro istanza all'interno di ogni client e sincronizzandoli tramite l'utilizzo dei reflector. Per comprendere meglio il funzionamento
dei passaggi successivi, si consiglia di immaginare questi dati come all'interno di un server fittizio a cui ogni client ha accesso come mostrato in figura~\ref{fig:CroquetServer.png}.\\
\img{CroquetServer.png}{In alto la vera struttura di Croquet, in basso una rappresentazione semplificata.}
Al contrario del model, ogni view va vista come una istanza di realtà aumentata unica per ogni utente. Ogni giocatore è immerso nella propria simulazione WebXR con un proprio 
\textit{engine}, una priopria telecamera e i propri oggetti di scena. Queste simulazioni interagiscono tra loro grazie agli eventi: all'interazione dell'utente con gli oggetti di scena,
la view invia un evento al model, che lo elabora, modifica il proprio stato e aggiorna le view degli altri utenti come mostrato in figura~\ref{fig:MultiUserBABYLON.png}.\\
\img[1]{MultiUserBABYLON.png}{Ogni utente ha la propria istanza di BabylonJS.}
\newline
Un ultimo aspetto dell'architettura da considerare è la creazione di una API che adattasse le classi \textit{general-purpose} di Croquet alle specifiche del progetto. Queste classi si pongono
in mezzo tra i model e le view utilizzate nell'elaborato e quelle di Croquet, modificando e riadattando le \textit{feature} già presenti nel framework e aggiungendone di nuove. 
Nello specifico si è deciso di implementare una \texttt{BaseView} ed un \texttt{BaseModel} che ereditassero rispettivamente da \texttt{Croquet.View} e \texttt{Croquet.Model}, dai 
quali poi estenderanno tutte le altre viste e i modelli del progetto.\\
Il \texttt{BaseModel} riporta le seguenti \textit{feature}:
\begin{itemize}
    \item \textbf{automatizzazione della parentela}: ogni modello creato con queste classe avrà un riferimento al modello padre;
    \item \textbf{creazione di un log personalizzato}: ogni modello avrà un log personalizzato con il proprio nome, utile per il debug;
    \item \textbf{gestione della distruzione}: ogni modello ascolterà l'evento \textit{`game-over'} avente come \textit{scope} il \textit{session ID} di modo che, in qualunque momento, in qualsiasi
    punto del codice, se il componente valuta che la partita sia terminata, può lanciare questo evento e distruggere tutti i modelli;
    \item \textbf{nuova inizializzazione}: ogni modello disporrà di due metodi da definire, \texttt{\_initialize} che fa le veci del vecchio \texttt{init}, ora utilizzato per il funzionamento
    del \texttt{BaseModel}, e \texttt{\_subscribeAll} in cui vanno inserite tutte le sottoscrizioni che quel modello deve effettuare.
\end{itemize}
Si noti che la separazione tra \texttt{\_initialize} e \texttt{\_subscribeAll} è solo a scopo di leggibilità del codice. Queste funzioni vengono chiamate entrambe all'avvio del modello, per
tanto non risulterebbe un problema se si facesse una sottoscrizione in \texttt{\_intialize} piuttosto che una inizializzazione di variabile in \texttt{\_subscribeAll}.\\
Si riporta il codice della classe \texttt{BaseModel} nel listato~\ref{lst:BaseModel.js}.
\code{BaseModel.js}{Classe \texttt{BaseModel}.}

Prima di procedere con l'elenco delle funzionalità della \texttt{BaseView} si vuole sottolineare che quando si parlerà di molteplici view, non si farà riferimento ad esse in senso
\textit{orizzontale}, ovvero che ogni view contraddistingua un utente diverso, bensì in senso \textit{verticale}, ovvero che ogni utente disponga di più view, ognuna delle quali mostra
oggetti diversi (si faccia riferimento alla figura\ref{fig:MultiViews.png}).\\
\img[0.6]{MultiViews.png}{Visione \textit{verticale} e \textit{orizzontale} delle view a confronto.}
Le feature che presenta la \texttt{BaseView} sono:
\begin{itemize}
    \item \textbf{automatizzazione della parentela}: ogni view creata con questa classe avrà un riferimento alla view padre;
    \item \textbf{automatizzazione del riferimento al model}: ogni vista creata con questa classe avrà un riferimento al model corrispettivo;
    \item \textbf{creazione di un log personalizzato}: ogni view avrà un log personalizzato con il proprio nome, utile per il debug;
    \item \textbf{nuova inizializzazione}: ogni view disporrà di tre metodi da definire, \texttt{\_initialize} che fa le veci del vecchio \texttt{init}, ora utilizzato per il funzionamento
    della \texttt{BaseView}, \texttt{\_subscribeAll} in cui vanno inserite tutte le sottoscrizioni che quella view deve effettuare e \texttt{\_initializeScene} che deve contenere tutte le
    inizializzazioni relative ad oggetti di scena;
    \item \textbf{automatizzazione dell'aggiornamento}: un punto cieco dell'architettura di Croquet è che la chiamata del metodo \texttt{update} non viene perpetrata da una view all'altra.
    Fornendo una lista \texttt{children} riempita con i riferimenti a tutte le view figlie, la \texttt{BaseView} potrà accedere a tutte le viste dalle quali poi richiamare i rispettivi
    \texttt{update} così da sopperire alla mancanza di Croquet. Viene fornito anche un metodo \texttt{\_update} da sovrascrivere per aggiungere funzionalità all'aggiornamento;
    \item \textbf{gestione della distruzione}: ogni view ascolterà l'evento \textit{`game-over'} avente come \textit{scope} il \textit{session ID} di modo che, se venisse lanciato, la view
    si distrugga automaticamente. Per attuare ciò viene lasciata una lista \texttt{sceneObjects} da riempire con tutte le \textit{mesh} di BabylonJS istanziate in modo tale che,
    alla chiamata di distruzione, la \texttt{BaseView} possa distruggere anch'essi. Inoltre, essendo una UI, si è previsto che le view avessero bisogno di un tempo d'attesa prima di 
    cancellare tutta la scena, affinchè ogni oggetto faccia la sua uscita di scena e/o mostri informazioni riguardanti il termine del gioco. Per realizzare ciò si è costruito un metodo 
    da estendere chiamato \texttt{\_endScene} nel quale effettuare tutte le animazioni del caso e che restituisca il numero di millisecondi da aspettare prima di lanciare il comando di 
    \texttt{detach};
    \item \textbf{informazioni condivise}: si è creata una struttura dati che fosse accessibile da tutte le viste e che racchiudesse le informazioni univoche per ogni partita come una sorta
    di \textit{singleton pattern}. Qui si possono trovare informazioni come il riferimento all'\textit{engine} di BabylonJS utilizzato, alla scena o alla telecamera.
\end{itemize}
Si noti che, per quanto possa essere richiamata da qualsiasi punto del codice, la \textit{`game-over'} venga lanciata solo dal modello. Questo perchè, essendo il modello l'unico 
componente che conosce lo stato della partita, è l'unico che può valutare se la partita sia terminata o meno.\\
Anche in questo caso, si lascia il sorgente della classe \texttt{BaseView} nel listato~\ref{lst:BaseView.js}.
\code{BaseView.js}{Classe \texttt{BaseView}.}
Fino ad ora si sono esposte le scelte progettuali effettuate per la realizzazione del progetto. Nella prossima sezione si vedranno più nel dettaglio le funzionalità implementate
nel progetto.

\section{Progettazione dettagliata}\label{sec:progettazione}
Qui verranno trattati dettagli implementativi come BaseModel e BaseView;
i vari scambi di eventi e messaggi tra i vari componenti etc

Cose da dire:
\begin{itemize}
\item Si è strutturata una rete di \texttt{Croquet.Model} che distribuisse ogni funzionalità del dominio ad un modello diverso, in modo da tenere separati i compiti e mantenere una
    gerarchia ordinata e intuitiva.

\item Lungo tutta la \textit{codebase} si è deciso di non utilizzare ereditarietà, preferendo la composizione in quanto ritenuta più flessibile e meno vincolante. La struttura si può
raffigurare comunque in una relazione padre-figlio in cui però non è il figlio a estendere il padre, bensì il padre a creare e contenere il figlio.\\

\item Definiti questi princìpi cardine si sono quindi costruite le fondamenta su cui basare il video game. La classe più importante, da cui poi si sviluppa tutta la struttura, è 
\texttt{GameModel}. Questa classe di fatto non gestisce una struttura dati ma fa da contenitore per tutti i modelli del gioco. Qui si trovano i riferimenti ai modelli dei giocatori,
del turno e del campo di battaglia.\\
Altra funzionalità importante per questa classe è di gestire le connessioni e i ruoli. All'avvio, in base al numero di partecipanti già presenti, questa classe avrà il compito di 
assegnare un ruolo al nuovo utente connesso scegliendo tra \textit{player 1}, \textit{player 2} e \textit{spettatore}. Inoltre, se un utente dovesse disconnettersi, questa classe
dovrà gestire una sua possibile riconnessione come anche prevedere una sequenza di terminazione nel caso in cui l'utente non dovesse riconnettersi.\\
Nella controparte \texttt{GameView} si possono trovare gli stessi riferimenti alle view corrispondenti dei modelli citati. Si noti che, dato che il model non contiene una struttura
dati, la \texttt{GameView} non crea alcun componente visibile all'utente, mantenendo coerenza con il principio di specularità tra model e view.\\

\item Il back-end nel progetto è rappresentato da un insieme di \texttt{Croquet.Model} che si occupano di gestire i dati e di fornire le funzionalità base per interagirci. Spesso vengono
anche utilizzate classi standard JavaScript contenute all'interno dei model che fungano da supporto per gestire le strutture dati. Si è deciso di non utilizzare ereditarietà 
all'interno del progetto, preferendo la composizione, in quanto si è ritenuto che non fosse necessario avere una gerarchia di classi. 
La prima classe model creata, nonchè quella passata 
\end{itemize}

\chapter*{Conclusioni}
\addcontentsline{toc}{chapter}{Conclusioni}
\markboth{CONCLUSIONI}{CONCLUSIONI}
I concetti fondamentali su cui si basa questo elaborato sono tre: XR, il videogame e le applicazioni distribuite.\\
\newline
In primo luogo, la combinazione delle varie tecnologie all'interno di un singolo progetto è considerabile un risultato di per sè. La sfida di integrare un sistema di distribuzione
come Croquet con un sistema di simulazione come WebXR ha portato a problematiche complesse e mai affrontate prima. Ancora prima della loro integrazione, c'è stata tutta una fase 
dedicata allo studio dei framework e alla scelta di quale fosse il più adatto per il progetto. Dopodichè è stato necessario esplorare operativamente le infrastrutture scelte 
applicandole ad un prototipo del sistema. Infine si è passati alla fase di integrazione vera e propria, che ha portato, tra le altre cose, problemi di comunicazione tra i vari 
componenti e di sincronizzazione tra i vari utenti.\\
Si vuole espandere quest'ultima problematica relativa alla sincronizzazione inquanto è stata una delle più complesse e che ha richiesto più tempo per essere risolta. Per venirne a 
capo sono state create strutture \textit{ad-hoc} come la \texttt{Root} e la \texttt{Game} in modo da poter gestire al meglio ogni caso di connessione, disconnessione e riconnessione 
che gli utenti potessero fare.\\
Un'ultima sfida che si è affrontata è stata cercare di rendere il sistema il più generico ed esetendibile possibile. Per quanto si sia cercato di lavorare in tal senso, tutt'ora vi sono
componenti, metodi e interazioni che avrebbero bisogno di essere riorganizzati e ristrutturati per rendere il progetto più modulare. Al netto di questo, il software fornisce comunque
dei pattern di base che possono essere utilizzati per creare applicazioni in realtà aumentata distribuite.\\
\newline
I piani futuri per questo elaborato sono molteplici. Si potrebbe pensare ad una standardizzazione delle classi di model e view e delle loro comunicazioni, in modo da poter creare
un framework che permetta di creare applicazioni di realtà aumentata distribuite in maniera più semplice. Si potrebbero revisionare ed espandere la struttura di \texttt{BaseModel} e
\texttt{BaseView}, in modo da fornire un'interfaccia più completa e più semplice da utilizzare.\\
Si è pensato anche alla continuazione del progetto per la conclusione delle \textit{feature} secondarie che lo riguardano come gli effetti delle carte, la possibilità di giocare
mostri tramite sequenze complicate tipo le evocazioni\footnote{Evocazione: nel gioco di carte questo termine è sinonimo di giocare sul campo un mostro.} speciali e le evocazioni 
\textit{Xyz}, e la possibilità di giocare magie \textit{terreno}\footnote{Magia terreno: carta magia che nella serie animata modificavano l'ambiente circostante.} che, nel contesto di
un applicativo XR, riscontrerebbero ancora più successo.\\	
\chapter*{Ringraziamenti}
\addcontentsline{toc}{chapter}{Ringraziamenti}
\markboth{RINGRAZIAMENTI}{RINGRAZIAMENTI}

Desidero esprimere la mia sincera gratitudine a tutte le persone che hanno contribuito in modo significativo al completamento di questa tesi di laurea. Il mio percorso di studio è 
stato arricchito dalla generosità, dal supporto e dalla fiducia di molti individui, ai quali vorrei dedicare questa sezione di ringraziamenti.\\
\newline
Innanzitutto, vorrei ringraziare il Professore Alessandro Ricci per avermi concesso l'opportunità di svolgere questa tesi sotto la sua guida esperta. La sua competenza e il suo 
sostegno costante sono stati fondamentali per la realizzazione di questo lavoro.\\
\newline
Un sentito ringraziamento va anche al corelatore della tesi, il Dottore Samuele Burattini, che ha dedicato tempo ed energie per seguirmi passo dopo passo nell'intero processo di
sviluppo della tesi. La sua pazienza e la sua disponibilità sono state inestimabili per il mio successo.\\
\newline
La mia famiglia merita un profondo riconoscimento per il loro costante appoggio e la fiducia incondizionata che mi hanno offerto durante il mio percorso di studi.
Senza il loro amore e il loro incoraggiamento, questo traguardo sarebbe stato molto più difficile da raggiungere.\\
\newline
Un ringraziamento speciale va ai miei amici universitari del gruppo \textit{Predapio}, con i quali ho condiviso gioie, sfide e momenti di studio. La vostra compagnia ha reso 
l'esperienza universitaria unica e indimenticabile.\\
\newline
Ai miei amici al di fuori dell'università, desidero esprimere la mia gratitudine per avermi sostenuto in ogni fase della mia vita accademica. La vostra amicizia è stata una fonte
di gioia e ispirazione.\\
\newline
Un grazie speciale va alla mia morosa Silvia, che ha sempre ascoltato pazientemente le mie discussioni sull'argomento della tesi, offrendo supporto emotivo e incoraggiamento
quando ne avevo più bisogno. La tua comprensione è stata un motore fondamentale per il mio successo.\\
\newline
In conclusione, questa tesi rappresenta un capitolo importante nella mia vita, e l'aiuto e il supporto di tutte queste persone sono stati fondamentali per raggiungere questo traguardo. 
Grazie di cuore a ciascuno di voi.






	
\backmatter	

\bibliography{bibliografia}{}
\bibliographystyle{plain}

\end{document}