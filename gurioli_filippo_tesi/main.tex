\documentclass[12pt,a4paper,twoside,openright]{book}

\usepackage[italian]{babel}
\usepackage{style/isi-lt}

\definecolor{dkgreen}{rgb}{0,0.6,0}
\definecolor{gray}{rgb}{0.5,0.5,0.5}
\definecolor{mauve}{rgb}{0.58,0,0.82}

\lstset{
  frame=single,
  captionpos=b,
  language=Java,
  aboveskip=3mm,
  belowskip=3mm,
  showstringspaces=false,
  columns=flexible,
  basicstyle={\small\ttfamily},
  numbers=none,
  numberstyle=\tiny\color{gray},
  keywordstyle=\color{blue},
  commentstyle=\color{dkgreen},
  stringstyle=\color{mauve},
  breaklines=true,
  breakatwhitespace=true,
  tabsize=3
}

\newcommand{\code}[2]{
    \lstinputlisting[
        caption={#2}, 
        label=lst:#1
    ]{code/#1}
}

\newcommand{\img}[3][0.8] {
    \begin{figure}
        \centering
        \includegraphics[width=#1\textwidth]{images/#2}
        \caption{#3}
        \label{fig:#2}
    \end{figure}
}

%--------------------------------------------------------------------
%---------------------- INFORMAZIONI SULLA TESI ---------------------
%--------------------------------------------------------------------

\universita{Alma Mater Studiorum -- Università di Bologna}
\campus{Campus di Cesena}
\dipartimento{Dipartimento di Informatica -- Scienza e Ingegneria}
\cdl{Corso di Laurea in Ingegneria e Scienze Informatiche}

\titolo{Come si progetta un gioco in realtà aumentata multiplayer usando webxr}
\materia{Sistemi embedded ed internet of things}

\laureando{Filippo Gurioli}

\relatore[Prof.]{Alessandro Ricci}
\correlatorea[Dott. Ing.]{Samuele Burattini}

\annoaccademico{2022 -- 2023}

\dedica{eventuale dedica o citazione...}

\makeindex

\begin{document}
\frontmatter 

\maketitle
\tableofcontents

\chapter{Introduzione}
\markboth{INTRODUZIONE}{INTRODUZIONE}

Questo progetto di tesi è stato sviluppato con l'obiettivo di esplorare le tecnologie più all'avanguardia nel settore della \textit{Extended Reality}. Si voleva produrre un software
grando di funzionare sul nuovo visore AR di casa Microsoft \textit{HoloLens 2} sfruttando le potenzialità offerte dalla macchina in concomitanza delle librerie più recenti
del settore.\\
\newline
Come caso di studio si è scelto un videogioco in modo tale da rendere più stimolante l'apprendimento e l'utilizzo delle tecnologie. Volendosi però concentrare più sull'architettura
del sistema piuttosto che sulle meccaniche di gioco, si è deciso un titolo semplice e conosciuto: \textit{Yu-Gi-Oh}. Il video game in questione rientra in nella branca dei giochi di
carte collezionabili, in cui due giocatori si sfidano a colpi di carte, ognuna con le proprie caratteristiche e abilità. Questo tipo di giochi si identificano in letteratura come
giochi di strategia \textit{turn-based}, caso di studio ideale per questo progetto di tesi.\\
\newline
La grande svolta che si voleva dare a questo progetto era quella di connettere più giocatori alla stessa partita. Anche in questo caso si può notare come \textit{Yu-Gi-Oh!} sia 
perfetto per questo scopo, prevedendo il minimo numero di giocatori per considerare il prodotto \textit{multiplayer} (due). Seppure il gioco sia \textit{turn-based}, si tenga presente
che si voleva realizzare un sistema \textit{real-time}, cioè che aggiornasse i vari utenti di ogni cambiamento in tempo reale. Questo tema assume un valore centrale lungo tutto lo
svolgimento della tesi, lungo la quale si potranno leggere le scelte attuate per realizzare un sistema di questo tipo.\\
\newline
Un'ultima caratteristica fondante di questo elaborato è la scelta di realizzarlo sul web. Dall'esplosione di Internet, il web è diventato il mezzo di comunicazione più utilizzato
al mondo. Questo ha portato a una continua evoluzione delle tecnologie che lo compongono, rendendolo sempre più potente e versatile. In particolare, la nascita di \textit{WebXR} ha
reso possibile di realizzare applicazioni di \textit{Extended Reality} direttamente sul web, senza la necessità di installare alcun software. Questo ha portato a una
sempre maggiore diffusione di applicazioni di questo tipo, rendendo il web un ambiente ideale per lo sviluppo di questo progetto.\\
\newline
La tesi si sviluppa in tre parti principali:
\begin{enumerate}
    \item \textbf{Definizione dei concetti di base}: in questa prima parte si definiscono i concetti di base che verranno utilizzati lungo tutto lo sviluppo del progetto. Questo 
    capitolo funge al lettore da introduzione a tutti i termini tecnici che verranno utilizzati nel proseguo della tesi. Si parte dalla definizione di \textit{Extended Reality} seguita
    dalla descrizione dei vari tipi di realtà che si possono incontrare. Si passa poi a descriverne la storia per poi terminare con l'esplorazione delle sue applicazioni
    più comuni (tra cui anche il \textit{gaming}).
    \item \textbf{Tecnologie}: nel secondo capitolo si descrivono le tecnologie utilizzate per lo sviluppo del progetto. Esso entra più nel merito del progetto, 
    introducendo framework e librerie utilizzate per la realizzazione del software. Si divide la trattazione in due macrocategorie: \textit{front end} e \textit{back end}. Nella prima
    si tratterà nello specifico di WebXR, Mixed Reality Toolkit e BabylonJS. Nella seconda si parlerà di NodeJS e Croquet.
    \item \textbf{Progetto}: l'ultimo capitolo è dedicato alla descrizione del progetto. Qui si cerca di chiarire come le tecnologie spiegate nel capitolo precedente siano state
    utilizzate e integrate all'interno del software. Si comincia con l'analisi dei requisiti, per poi passare allo studio del design architetturale e finire con la progettazione e
    l'implementazione vera e propria.
\end{enumerate}
Nei prossimi capitoli di questo documento si approfondiranno i moduli sopra menzionati, cercando di spiegare nel dettaglio il contesto del progetto e le scelte effettuate per
realizzarlo. 
	
\mainmatter

\pagestyle{fancy} 
\fancyhead[LE,RO]{\thepage}
\fancyfoot{}

\chapter{Stato dell'arte}\label{chap:Letteratura}
Di seguito si riporteranno informazioni circa lo stato dell'arte riguardante la Extended Reality, in particolare, si toccherà il tema dello sviluppo di videogiochi in ambito 
Augmented Reality e Virtual Reality. Successivamente si analizzerà la letteratura riguardante il tema del multiplayer e come questo sia attualmente realizzato nei software comuni,
sottolineando l'importanza del web al giorno d'oggi.

\section{Extended Reality}\label{sec:XR}
    La Extended Reality è un campo molto vasto che comprende diverse tecnologie ed è in continua evoluzione. Talvolta non è possibile fornire una definizione univoca dei vari 
    concetti per via delle diverse visioni che gli esperti in materia hanno. Lo scopo della sezione seguente è quindi quello di chiarire i concetti fondamentali riguardanti
    questa tecnologia, esplorare le diverse accezioni e presentare alcuni esempi di utilizzo.
    \subsection{Definizione}\label{subsec:XRDef}
        Con il termine Extended Reality (abbreviato in XR) si intende un insieme di tecnologie che permettono di estendere la realtà, ovvero di aggiungere informazioni al mondo 
        reale. Sotto questa macro definizione rientrano le tecnologie di realtà aumentata (AR), realtà virtuale (VR) e realtà mista (MR). Per AR si intende la tecnologia che permette 
        di sovrapporre informazioni digitali al mondo reale, mentre per VR si intende la tecnologia che permette di immergere l'utente in un ambiente virtuale. La MR è una tecnologia
        che permette di conciliare il mondo reale con quello virtuale in modo totalmente trasparente per l'utente. Come ogni calcolatore, queste tecnologie prevedono dispositivi 
        di input e di output per l'interazione tra macchina e utente. Venendo completamente immersi in queste realtà virtuali, l'utilizzo di una tastiera o un mouse risulterebbe
        inefficace, si sono quindi progettati dispositivi ad hoc per l'interazione (input) e per la ricezione delle informazioni (output). Per quanto riguarda l'input, si possono
        utilizzare dei controller, dei guanti o il proprio corpo, mentre per l'output domina l'utilizzo di visori. Questi dispositivi sono in grado di rilevare i movimenti 
        dell'utente e di trasmetterli al sistema, che li elabora e li utilizza per modificare l'ambiente virtuale. Questo ambiente virtuale può essere visualizzato dall'utente 
        tramite un visore, che può essere un visore VR, un visore AR o un visore MR.
        \paragraph{Augmented Reality e Mixed Reality} sono due concetti molto simili, dei quali si trovano definizioni ambigue, talvolta discordanti. La definizione più comunemente
        accettata per Augmented Reality è quella fornita da Azuma\cite{Azuma1997}:
            \begin{quote}
                \textit{AR systems have the following three characteristics: (1) combine real and virtual, (2) are interactive in real time, and (3) are registered in 3D.}
            \end{quote}
            Si noti che non è specificato nessun tipo di dispositivo di output, potrebbero essere  AR anche dispositivi che forniscono informazioni tattili, gustative o olfattive,
            benchè la loro implementazione è di fatto un settore tutt'ora nascente (motivo per il quale nella sezione precedente non li si è menzionati). La definizione formale di 
            Mixed Reality è invece fornita da Milgram\cite{Milgram1994}:
            \begin{quote}
                \textit{A mixed reality (MR) system is one that combines real and virtual environments seamlessly.}
            \end{quote}
            Questa definizione è molto simile a quella di Azuma, ma non è specificato il fatto che l'ambiente virtuale debba essere tridimensionale. La definizione di Milgram non
            specifica, inoltre, che l'ambiente virtuale debba essere sovrapposto a quello reale, ma che i due ambienti debbano essere combinati in modo da risultare indistinguibili. 
            In quest'ottica si può quindi affermare che la AR è un caso particolare di MR, in cui l'ambiente virtuale è sovrapposto a quello reale.\\
            Il libro di Schmalstieg e Höllerer \textit{Augmented Reality: Principles and Practice}\cite{Schmalstieg2016} fornisce una schematizzazione della gerarchia tra AR e MR
            mostrata in figura\ref{fig:ARvsMR} che permette di comprendere meglio la relazione tra le due tecnologie. 
            \img{ARvsMR}{Gerarchia tra AR e MR - \textit{`The mixed reality continuum'}.}
            
    \subsection{Storiografia}\label{subsec:XRTipologie}
        Come già accennato prima, il mondo della Extended Reality non si ferma al solo utilizzo di visori. Nella storia esistono esempi di applicazioni di AR e VR molto particolari 
        e interessanti. Si presenteranno di seguito alcuni di questi esempi, cercando di esplorare le scelte architetturali più interessanti.
        \paragraph{La spada di Damocle} (figura\ref{fig:SpadaDiDamocle}) è considerato  il primo prototipo di visore VR mai costruito nella storia e, come tale, ha diritto di essere
            citato come primo elemento di questa sezione. Il visore è stato costruito da Ivan Sutherland nel 1968 ed era pesante al punto da dover essere fissato al soffito per
            essere sorretto, da questo il suo nome. Il visore era in grado di mostrare all'utente un cubo tridimensionale, che poteva essere osservato da diverse angolazioni 
            semplicemente muovendo la testa, era infatti già presente, seppur in forma molto rudimentale, il concetto di \textit{head tracking}. Il visore era inoltre dotato di un 
            controller, che permetteva di interagire con l'ambiente virtuale. Questo prototipo è stato il primo passo verso la realizzazione di visori VR, ma non è stato
            mai commercializzato.
            \img{SpadaDiDamocle}{La spada di Damocle, primo prototipo di visore VR.}

        \paragraph{Sensorama} (figura\ref{fig:Sensorama}) è un dispositivo che permette di immergere l'utente in un
            ambiente virtuale, mostrando un video stereoscopico a 3 dimensioni e riproducendo suoni e odori. Il dispositivo è stato
            progettato per essere utilizzato in sale cinematografiche, in modo da coinvolgere lo spettatore con tutti e 5 i sensi.
            La macchina di Morton Heilig era però troppo costosa, e questo, unito alle dimensioni della macchina e alla scarsa qualità
            dei video, ha portato al fallimento del progetto.
            \img{Sensorama}{Sensorama, dispositivo fornito di: immagini stereo 3D, vibrazioni, vento, emettitore di odori e audio stereo.\cite{wikiSensorama}}

        \paragraph{IllumiRoom} è una tecnologia Microsoft che punta a creare un'esperienza AR estendendo la realtà di gioco
            normale, composta dallo schermo e dal controller, all'intera stanza. Questa tecnologia è stata presentata nel 2013
            e permette di proiettare sulle pareti della stanza in cui si gioca le immagini del gioco, in modo da estendere
            l'ambiente virtuale a quello reale. Questa tecnologia è stata sviluppata per la console Xbox One, ma non è mai
            stata commercializzata.\cite{Schmalstieg2016}
            \img{IllumiRoom.png}{IllumiRoom, prodotto capace di estendere l'ambiente di gioco all'intera stanza.}
        
        \paragraph{L'HTC Vive} è uno dei più famosi visori VR, è stato sviluppato da HTC e Valve Corporation ed è
            rilasciato nel 2016. L' \textit{Head Mounted Display} (HDM) è dotato di due schermi OLED da $1080\times1200$ pixel, 
            con un refresh rate di 90Hz\cite{Vive}. Questo visore è uno dei più conosciuti sul mercato, 
            e tra le altre cose, deve anche la sua fama alla tecnologia di tracciamento dei movimenti chiamata Lighthouse, che permette di tracciare
            i movimenti dell'utente in modo preciso e affidabile. Questa tecnologia è stata sviluppata dalla Valve
            e consiste di due stazioni base che emettono dei segnali infrarossi che vengono rilevati da sensori posti 
            sulle periferiche. Queste stazioni base vengono poste alle estremità della stanza, in modo tale da circoscrivere
            un'area entro la quale l'utente è libero di muoversi.
            \img{HTCVive.jpg}{HTC Vive, da sinistra a destra: stazioni base, controller e visore.}

        \paragraph{I dispositivi mobili} sono alla base della maggior parte delle applicazioni di AR. Questi dispositivi
            sono dotati di fotocamera, accelerometro, giroscopio, magnetometro e geolocalizzazione che permettono di rilevare i movimenti
            dell'utente e di orientare la fotocamera. Questi dispositivi sono inoltre dotati di schermo, che permette di
            visualizzare l'ambiente virtuale. I cellulari sono molto utilizzati per applicazioni di Augmented Reality, in quanto
            sono dispositivi che tutti possiedono e che sono in grado di fornire un'esperienza soddisfacente.
            Attualmente sul mercato è esploso il fenomeno degli \textit{AR advertising}, ovvero applicazioni che permettono
            di visualizzare i prodotti venduti dalle aziende in modo virtuale direttamente nello spazio reale. Queste applicazioni
            sono molto utilizzate per la vendita di mobili, in quanto permettono di proiettare il prodotto in modo realistico
            e di valutare se questo si adatta all'ambiente in cui si vuole inserire. Anche il settore videoludico non ha 
            perso l'occasione di sfruttare questa tecnologia, infatti sono state sviluppate applicazioni che permettono di
            visualizzare i personaggi dei propri giochi preferiti direttamente nella propria stanza. Un esempio di questo
            tipo di applicazioni è \textit{Pokémon GO}, gioco sviluppato da Niantic nel 2016, che permette di catturare i
            Pokémon direttamente nella propria città.
            \img{PokemonGo.png}{Pokémon GO, applicazione di AR per dispositivi mobili.}
            
        \paragraph{Hololens} è il dispositivo Microsoft rilasciato nel 2016 che ha portato il mondo delle tecnologie 
            AR e MR a un livello superiore. Questo HMD è dotato di un processore di fascia alta, una GPU per il calcolo
            di immagini ed una Holographic Processing Unit (HPU) che permette di elaborare i dati provenienti dai sensori
            in tempo reale. Il visore monta anche una telecamera di profondità, sensori avanzati per il tracciamento dei
            movimenti e un sistema di rilevamento del suono. Questo visore è in grado di proiettare immagini in modo
            realistico sulle lenti con display ottico 3D, in modo da sembrare che l'immagine sia davanti all'utente.
            Il visore ha un costo ancora troppo elevato per essere utilizzato da un utente comune, ma è stato utilizzato
            in ambito industriale per la progettazione di automobili e per la formazione di chirurghi. \\
            Tra le varie feature che il visore offre ci sono:
            \begin{itemize}
                \item \textbf{Spatial Mapping}: permette di mappare l'ambiente circostante e di rilevare gli oggetti
                    presenti, in modo da poterli utilizzare per interagire con l'ambiente virtuale.
                \item \textbf{Spatial Sound}: permette di rilevare la posizione dell'utente e di modificare il suono in base
                    ad essa.
                \item \textbf{Gaze Tracking}: permette di rilevare la direzione dello sguardo dell'utente.
                \item \textbf{Voice Recognition}: permette di rilevare i comandi vocali dell'utente.
                \item \textbf{Gesture Recognition}: permette di rilevare i movimenti delle mani dell'utente svincolando
                    l'utente dall'uso di controller e guanti (un ulteriore passo avanti verso la \textit{computazione ubiqua}
                    \footnote{Computazione ubiqua: modello post-desktop di interazione uomo-macchina, in cui l'elaborazione
                    delle informazioni è stata interamente integrata all'interno di oggetti e attività di tutti i giorni.\cite{ubiCompWiki}}).
            \end{itemize}
            In conclusione si noti che ogni sensore è completamente integrato nel visore, che non necessita quindi di
            dispositivi esterni rendendolo completamente wireless.
            \img{Hololens.png}{Hololens, visore AR/MR di Microsoft.}

    \subsection{Applicazioni}\label{subsec:XRapplicazioni}
        Le applicazioni di Extended Reality sono molteplici e vanno dai videogiochi, all'industria, alla medicina. In questa sezione si analizzeranno alcune applicazioni di AR e VR, 
        cercando di capire quali sono le scelte architetturali più interessanti.
        \paragraph{L'industria} sta utilizzando queste nuove tecnologie per migliorare la produttività e la qualità dei prodotti. Un esempio di questo utilizzo è quello già
            accennato di AR Ads per la vendita di mobili, vestiti e altre merci, di cui i principali esempi sono:
            \begin{itemize}
                \item American Apparel che ha sviluppato un'applicazione in grado di mostrare i vestiti in modo realistico sul proprio corpo;
                \item IKEA che ha sviluppato un'applicazione che permette di visualizzare i propri mobili direttamente nella propria casa;
                \item De Beers che ha sviluppato un'applicazione che permette di provare gli anelli direttamente sul proprio dito.
            \end{itemize} 
            Un altro esempio di utilizzo di AR in ambito industriale è quello di \textit{Remote Assistance}, ovvero l'assistenza remota. Questa tecnologia permette di fornire 
            assistenza a distanza, dando la possibilità di vedere ologrammi illustrativi direttamente nello spazio reale, facilitando il lavoro dei tecnici e permettendo di 
            ridurre i costi di formazione.\\

        \paragraph{Augmented Medicine} è il termine che è stato coniato per indicare l'utilizzo di dispositivi AR in ambito medico. Tra queste applicazioni si possono citare:
            \begin{itemize}
                \item \textbf{Formazione di chirurghi}: in quanto permette di visualizzare in modo realistico gli organi 
                    interni del paziente, permettendo di eseguire operazioni in modo virtuale e di simulare interventi chirurgici.
                \item \textbf{Progettazione di protesi}: in quanto permette di visualizzare in modo realistico le protesi 
                    direttamente sul paziente, permettendo di valutare se la protesi si adatta al corpo del paziente.
                \item \textbf{Vista in profondità}: l'AR ha permesso a chirurghi di vedere in tempo reale tumori, fratture e aneurismi direttamente sul paziente, in modo da poter 
                    agire in modo più preciso e sicuro.
            \end{itemize}
            \img{ARMedicine.png}{Esempio di applicazione di AR in ambito medico.}
        \paragraph{Videogiochi}

\section{Videogiochi}\label{sec:Videogiochi}
In questa sezione si parlerà di videogiochi, in particolare si analizzeranno le varie tipologie di Videogiochi
e si mostreranno alcuni esempi di applicazioni di queste tipologie.
    \subsection{Definizione}\label{subsec:VideogiochiDef}
    \subsection{Applicazioni}\label{subsec:VideogiochiApplicazioni}

\section{Multiplayer}\label{sec:Multiplayer}
    \subsection{Definizione}\label{subsec:MultiDef}
    \subsection{Tipologie}\label{subsec:MultiTipologie}
    Qui si parlerà delle varie tipologie di multiplayer, facendo riferimento anche al libro Multiplayer Game Programming.
    \subsection{Applicazioni}\label{subsec:CoopApplicazioni}
        \paragraph{Al giorno d'oggi}\label{par:MultiOggi}
        \paragraph{Web}\label{par:CoopWeb}

%\chapter{Progetto}\label{chap:Progetto}
Questo capitolo si pone l'obiettivo di introdurre il lettore agli argomenti trattati nei capitoli successivi, dove vengono affrontati dettagli tecnici del progetto. Qui si esporranno
in maniera generale l'idea alla base del progetto, le sue sfide principali e il traguardo che si pone di conseguire.\\
\newline
Il software che si è sviluppato consiste in una rivisitazione del gioco \textit{Yu-Gi-Oh!} trasportato in realtà aumentata. \textit{Yu-Gi-Oh!} è un gioco di 
carte collezionabili in cui due giocatori si sfidano in un duello nel quale devono cercare di ridurre i punti vita dell'avversario a zero. Per fare ciò, i duellanti sono dotati di un
mazzo di carte composto da mostri, magie e trappole. I mostri rappresentano delle truppe schierabili dalla parte del giocatore che possono attaccare l'avversario o difendere il
giocatore stesso, mentre le magie e le trappole sono carte che possono influenzare lo svolgimento del gioco tramite i loro effetti. Il gioco è a turni e ogni turno è diviso in fasi. Le 
fasi contraddistinguono le azioni che il giocatore può intraprendere durante quel periodo di tempo.\\
\newline
Esiste anche una serie animata basata su questo gioco di carte ed è proprio da questa che nasce l'idea del progetto. In questa serie, quando il protagonista gioca una carta mostro, 
viene proiettato un ologramma del mostro stesso che si materializza sul campo di battaglia. L'ologramma è visibile a tutti i giocatori e può essere interagito da chiunque.\\
Venuti a conoscenza dell'esistenza di un HMD come HoloLens, si è pensato di realizzare uno \textit{strategy-game} in realtà aumentata e \textit{real-time} che permettesse di riprodurre 
quanto più fedelmente l'esperienza riportata nella serie.
\newline
Il progetto consiste quindi nella realizzazione di un video game distribuito, ovvero un gioco che permetta a più utenti di interagire tra loro in un ambiente virtuale condiviso. 
Si è posto sin da subito l'obiettivo di realizzare il gioco tramite una \textit{web app}, ovvero un'applicazione web
che può essere eseguita da un browser. In questo modo non ci sarebbero stati problemi di portabilità, rendendolo accessibile a chiunque. In aggiunta, 
lo sviluppo web è ormai dominante nel settore informatico e permette di raggiungere un pubblico più vasto oltre che comunità più grandi e attive.\\
La tecnologia più utilizzata in ambito web per la realizzazione di applicazioni AR è WebXR. WebXR permette di creare realtà aumentate eseguibili su browser in cui ogni utente
vive la stessa esperienza ma in maniera indipendente. Bisognava quindi ancora trovare uno strumento per sincronizzare i vari utenti in modo da condividere l'esperienza. A supporto 
di ciò si è utilizzato Croquet, framework che sopperisce esattamente il requisito posto. Si noti che tutta la sezione riguardante il supporto all'esperienza distribuita è un tema
non trattato nel corso di studi e che è stato affrontato autonomamente. Questo ha portato però a dei tagli sulla realizzazione delle funzionalità di gioco, che sono state ridotte
al minimo indispensabile per poter concentrare gli sforzi sullo sviluppo dell'architettura generale del sistema.\\
\newline
Per conoscere nel dettaglio quali tecnologie si è utilizzato, come funzionano e perchè sono state scelte si rimanda al capitolo \ref{chap:Tecnologie}.
\chapter{Tecnologie}\label{chap:Tecnologie}
L'obiettivo del seguente capitolo è introdurre le tecnologie software utilizzate per lo sviluppo del progetto. Si parlerà dei principali framework attualmente utilizzati per sviluppare
applicazioni XR, di quelli utilizzati per lo sviluppo di applicazioni web e di quelli impiegati per lo sviluppo di applicazioni condivise. Durante la trattazione saranno fatti anche
confronti tra framework simili motivando la scelta intrapresa per questo progetto, analizzando anche i vari aspetti che hanno portato a questa decisione.

\section{MRTK}\label{sec:MRTK}
MRTK (acronimo per Mixed Reality Toolkit) è un framework open source sviluppato da Microsoft per Unity che permette di sviluppare applicazioni per HoloLens e Windows Mixed Reality\cite{MRTK}.\\
Dopo la nascita del dispositivo HoloLens mancavano ancora tutta una serie di API che permettessero al programmatore di svincolarsi dalle basi tecniche, principalmente legate 
alla gestione degli input e della spazialità, e di concentrarsi sulla creazione di applicazioni.
Questo framework è stato quindi sviluppato per creare un insieme di componenti riutilizzabili, che permettano di sviluppare applicazioni per la realtà mista in modo più semplice e 
veloce. Le principali componenti sono:
\begin{itemize}
    \item \textit{boundary system}, utilizzato nelle applicazioni VR per definire il confine fisico in cui l'utente può muoversi;
    \item \textit{camera system}, utilizzato per gestire la camera in modo efficiente in entrambe le simulazioni AR e VR;
    \item \textit{input system}, utilizzato per gestire ogni tipo di input utente, dal touchscreen ai controller, dal mouse e tastiera all'\textit{air tap};
    \item \textit{scene system}, creato per gestire le varie scene Unity e come queste interagiscono tra loro nel contesto XR;
    \item \textit{spatial awareness system}, utilizzato per gestire la posizione e l'orientamento dell'utente all'interno della scena;
    \item \textit{rendering system}, utilizzato nel rendering della scena virtuale, sfruttando funzioni avanzate come \textit{clipping}, \textit{pulse shaders} e \textit{materials}.
\end{itemize}
Il framework è stato sviluppato per essere utilizzato con Unity, un motore di gioco multipiattaforma che permette di creare applicazioni 2D e 3D. In quanto tale, supporta tutta
l'infrastruttura di Unity dei prefab, dei componenti e del game loop. Questi non verranno approfonditi per non allontanarsi troppo dallo scopo di questo lavoro tuttavia, per 
approfondire, si faccia riferimento al testo \textit{`Unity in Action'} di Joe Hocking\cite{hocking2018}. Essendo 
l'applicazione del progetto sviluppata per HoloLens 2 (di casa Microsoft), MRTK è stata quasi una scelta obbligata per poter creare ologrammi in modo nativo per questo dispositivo.\\
Si noti che il Mixed Reality Toolkit è stato prodotto in supporto del motore di gioco Unity ma è stato recentemente portato anche per Unreal Engine 4, BabylonJS e OpenXR. Nello 
specifico, BabylonJS è stata una delle tecnologie utilizzate nella tesi, motivo per il quale, nella sezione~\ref{sec:BabylonJS}, verrà approfondito il porting per BabylonJS.

\section{WebXR}\label{sec:WebXR}
WebXR è un insieme di specifiche web che permettono di creare applicazioni XR per il web. Questo permette di creare applicazioni che possono essere eseguite su qualsiasi dispositivo
che disponga di un browser, senza dover installare alcuna applicazione\cite{WebXR}.\\
WebXR crea una struttura capace di interagire con tutta la gamma di dispositivi XR, creando una \textit{session mode} diversa in base a quale dispositivo si collega. 
In base al sistema collegato vengono anche rese accessibili le funzionalità specifiche di quell'hardware.\\
Questa tecnologia permette anche una gestione capillare della sessione, organizzando i vari eventi che possono verificarsi durante l'esecuzione dell'applicazione, gestendo tutti
gli untenti collegati e fornendo anche un sistema per la gestione della disconnessione.\\
WebXR fornisce anche delle API per la creazione di un \textit{XR loop}, ovvero un ciclo di esecuzione che permette di gestire la scena virtuale e la sua interazione con l'utente.\\
In ultima istanza, questa tecnologia permette di gestire tutte le informazioni riguardanti la spazialità, ovvero la posizione e l'orientamento dell'utente, di gestire l'input 
dell'utente e di gestire la visualizzazione della scena virtuale con funzionalità avanzate come il \textit{XREye}, le \textit{primary} e \textit{secondary views} e la \textit{viewport}.
\newline \newline
In questo progetto le API fornite da WebXR non sono state usate direttamente, nonostante ciò era importante conoscere questa tecnologia per poter comprendere il funzionamento
di BabylonJS, che è stato utilizzato per la creazione dell'applicazione web. Si noti infine che WebXR è una tecnologia che garantisce un alto grado di sicurezza, in quanto
l'accesso alle pagine che la utilizzano è permesso solo tramite il protocollo HTTPS.

\section{NodeJS}\label{sec:NodeJS}
NodeJS è un runtime JavaScript open source che permette di eseguire codice JavaScript al di fuori di un browser\cite{NodeJS}. Un runtime JavaScript è un ambiente di esecuzione
in cui il codice JavaScript viene eseguito, in questo caso, NodeJS permette di eseguire codice JavaScript su un server.\\
NodeJS è stato sviluppato per essere utilizzato con il motore JavaScript V8 di Google, che è anche il motore JavaScript utilizzato da Google Chrome. Questo permette di utilizzare
le stesse funzionalità JavaScript sia lato client che lato server, creando un ambiente di sviluppo omogeneo e omnicomprensivo di tutte le funzionalità. Le feature più importanti
di NodeJS sono due, la gestione degli eventi asincrona e il \textit{Node Package Manager} o NPM.
\begin{itemize}
    \item \textit{Gestione degli eventi asincrona}: NodeJS è stato sviluppato per gestire un elevato numero di connessioni simultanee, per questo motivo è stato progettato con un
    modello di gestione degli eventi asincrono. Questo permette di gestire le richieste in arrivo senza dover creare un thread per ognuna di queste.
    Questa tecnica è anche utilizzata per la gestione degli I/O in modo tale da non bloccare l'unico thread disponibile.
    \item \textit{NPM}: NPM è un gestore di pacchetti per NodeJS che permette di installare e gestire le dipendenze di un progetto. Questo permette di creare un ecosistema di
    librerie e framework che possono essere utilizzati per lo sviluppo di applicazioni web. NPM è il più grande ecosistema di librerie open source al mondo, con oltre 1 milione
    di pacchetti disponibili.
\end{itemize}
L'\textit{event loop} di NodeJS è il cuore di questo runtime e fulcro della sua grande diffusione. Si riporta la sua descrizione dal sito ufficiale\cite{NodeJS}:
\begin{quote}
    [...] In altri sistemi, c'è sempre una chiamata bloccante per avviare il ciclo degli eventi. Tipicamente, il comportamento è definito attraverso i callback all'inizio di uno script, 
    e alla fine un server viene avviato attraverso una chiamata bloccante come \texttt{EventMachine::run()}. In Node.js, non c'è una tale chiamata per avviare il ciclo degli eventi. Node.js 
    entra semplicemente nel ciclo degli eventi dopo l'esecuzione dello script di input. Node.js esce dal ciclo degli eventi quando non ci sono più callback da eseguire.
\end{quote}
Nel progetto di questa tesi NodeJS è stato utilizzato per creare un server in un indirizzo IP locale su cui utilizzare il protocollo HTTPS affinchè le funzionalità di WebXR fossero
accessibili. Il server avrà il compito anche di eseguire l'applicazione web creata con Croquet.

\section{BabylonJS}\label{sec:BabylonJS}
BabylonJS è un framework open source per la creazione di applicazioni web 3D e XR\cite{BabylonJS}.\\
La scelta di BabylonJS, come tutte le principali feature di questo framework, sono un argomento molto ampio di questa tesi, motivo per il quale si analizzeranno una alla volta di seguito.

\subsection{Motivazioni e principali concorrenti}\label{subsec:BabylonJS_motivazioni}
La scelta di BabylonJS è stata dettata da una serie di motivazioni. In primo luogo, come già accennato, questo framework integra (anche se solo in parte) il Mixed Reality
Toolkit, dando accesso a molti componenti comuni a questo framework come \textit{Holographic Button}, \textit{Near Menu}, \textit{Hand Menu}, \textit{Holographic Slate} e altri 
ancora, incorporando anche molte funzionalità come il \textit{follow behaviour}, l'\textit{eye tracking} e l'\textit{hand tracking}.\\
In secondo luogo, BabylonJS supporta compleatamente WebXR, facendo da intermediario tra le API fornite da WebXR e il programmatore. Questo permette di creare applicazioni web
che possono essere eseguite su qualsiasi dispositivo XR, tramite l'utilizzo di interfacce semplici e intuitive.\\
In ultima istanza BabylonJS è stato scelto per la vasta gamma di funzionalità native che offre, insieme alla valida e completa documentazione relativa alle funzionalità. BabylonJS
può essere definito a tutti gli effetti un motore di gioco, in quanto fornisce un game loop, la gestione delle scene, delle luci, delle ombre, dei materiali, delle mesh e 
delle animazioni.
\newline \newline
I principali concorrenti di BabylonJS sono due: A-Frame e Unity. A-Frame si appoggia su ThreeJS che è un framework per la creazione di applicazioni web 3D. Per capire il motivo per
cui si è preferito scegliere BabylonJS bisogna prima sottolineare che ogni framework utilizzato nel videogioco è in JavaScript, questo permette di creare un ecosistema di librerie
e framework che possono essere utilizzati in modo omogeneo. ThreeJS è un framework scritto in JavaScript, ma A-Frame è scritto in HTML, questo crea un problema di omogeneità
e di gestione delle dipendenze che, seppur gestibile, non è ottimale. A questo bisogna anche aggiungere che BabylonJS è un framework più completo e più performante di A-Frame che,
tra le altre cose, non supporta MRTK.\\
Unity è forse il \textit{game engine} più famoso e utilizzato al giorno d'oggi, questo permette di creare applicazioni 2D e 3D per qualsiasi piattaforma. È stato valutato come
alternativa a BabylonJS per la creazione dell'applicazione web, ma in conclusione si è deciso di scartarlo per due motivi. In primo luogo, Unity, essendo scritto in C\#, avrebbe
comportato non poche difficoltà nell'integrazione con il resto della codebase. In aggiunta a questo, Unity non supporta nativamente WebXR, seppur esistano \textit{plug-in} che permettono
di utilizzare WebXR, questi sono comunque meno performanti e meno stabili rispetto a BabylonJS.\\

\subsection{Caratteristiche principali}\label{subsec:BabylonJS_caratteristiche}
Le caratteristiche principali che BabylonJS offre, in relazione anche al loro impiego nell'elaborato, sono:
\begin{itemize}
    \item \textit{WebXR}: come già accennato, BabylonJS supporta nativamente WebXR, permettendo di creare applicazioni web che possono essere eseguite su qualsiasi dispositivo XR, 
    accedendo semplicemente ad un indirizzo web. 
    \item \textit{MRTK}: BabylonJS supporta MRTK, permettendo di utilizzare molti i componenti di questo framework. Come già dichiarato, i più importanti sono
    bottoni, menù e pannelli. Insieme a questi componenti vengono fornite tutta una serie di funzionalità come il \textit{follow behaviour}, ovvero la capacità di un oggetto di 
    seguire l'orientamento della telecamera, rimanendo sempre visibile per l'utente, l'\textit{hand tracking}, ovvero la capacità di riconoscere le mani dell'utente e di 
    carpire le sue interazioni, e l'\textit{hand constraint behaviour}, ovvero la capacità di un oggetto di seguire il movimento della mano dell'utente, affinchè l'utente possa
    usufruire dell'oggetto ogni qual volta rivolga il suo sguardo sulle proprie mani.
    \item \textit{game engine}: BabylonJS è un motore di gioco completo, questo fornisce un insieme di funzionalità che permettono di creare applicazioni complesse molto più rapidamente.
    Tra le funzionalità più importanti si possono citare la creazione di un \textit{engine} per il rendering delle scene, la gestione delle varie scene, la gestione delle telecamere,
    la presenza di un motore fisico, la gestione di luci, ombre, materiali, mesh e animazioni, la gestione di eventi e la gestione di input.
    \begin{itemize}
        \item L'engine deve essere unico per tutta l'applicazione, fornisce utility di base come gli FPS, la dimensione della finestra e la possibilità di creare un ciclo di rendering.
        \item Le scene sono l'elemento principale di BabylonJS, ogni scena è composta da una serie di oggetti che possono essere mesh, luci, telecamere, materiali, ecc. Ogni scena
        può essere vista come un livello di un videogioco, in quanto può essere caricata e scaricata in qualsiasi momento.
        \item Le telecamere sono un componente che permette di visualizzare la scena, BabylonJS supporta diversi tipi di telecamere come la telecamera prospettica, la telecamera ortografica,
        la telecamera sferica e la telecamera cilindrica.
        \item Il motore fisico è un componente che permette di gestire la fisica all'interno della scena, permettendo di gestire la gravità, le collisioni e le forze.
        \item Le luci sono un componente che permette di illuminare la scena, BabylonJS supporta diversi tipi di luci come la luce direzionale, la luce puntiforme, la luce spot e la luce
        emisferica.
        \item Gli eventi rappresentano un qualsiasi avvenimento all'interno della scena, esiste una rete di \textit{subject-observer} creata ad hoc per osservare e gestire ogni tipo
        di evento che possa verificarsi nella scena.
        \item Gli input rappresentano una qualsiasi interazione dell'utente con la scena, BabylonJS supporta diversi tipi di input e li gestisce con eventi specifici.
    \end{itemize}
\end{itemize}

\section{Croquet}\label{sec:Croquet}
\chapter{Caso di studio: un gioco di carte competitivo in WebXR}\label{chap:Sviluppo}
In questo capitolo si affronteranno gli aspetti implementativi del progetto. Si inizierà con l'analisi dei requisiti, per poi passare alla fase di studio del design architetturale e
finire con la progettazione e l'implementazione vera e propria.

\section{Analisi dei requisiti}\label{sec:Analisi}
    L’obiettivo del progetto è creare un’ambiente di realtà aumentata condivisa
    in cui l’utente possa giocare contro un altro al gioco di carte Yu-Gi-Oh. L’esperienza che il giocatore proverà dovrà essere quanto più simile alla versione
    proposta nella serie animata omonima.
    \newline \newline
    Al momento dell’avvio l’utente dovrà affrontare un duello contro un’altra persona a Yu-Gi-Oh. Per la decisione del regolamento da seguire si è optato per
    una versione semplificata del gioco. Il giocatore potrà giocare carte mostro che
    rappresentano delle truppe schierate dalla parte del possessore. Queste truppe
    potranno quindi attaccare l’avversario per ridurne i punti vita. Saranno presenti anche carte magia e trappola che, tra i vari effetti, potranno modificare
    i punti vita, l’ambiete di gioco in cui gli utenti giocano o anche l’attacco e
    la difesa dei mostri propri e avversari. L’obiettivo del gioco consiste quindi
    nell’azzerare i punti vita dell’avversario, che comporterà la conclusione della
    simulazione.

    \subsection{Requisiti funzionali}\label{subsec:requisitiFunzionali}
        \begin{itemize}
            \item Il giocatore sarà in grado di vedere gli ologrammi personali e condivisi in tempo reale;
            \item il giocatore potrà interagire con un mazzo di carte virtuale pescando la prima carta;
            \item il giocatore potrà posizionare le carte che ha in mano sul campo e di
            conseguenza far apparire la corrispondente carta nello spazio di gioco
            condiviso;
            \item il giocatore potrà avanzare di fase, come ordinare l'attacco di un mostro tramite un menù apposito;
            \item ad ogni danno (o cura) inflitto (o subito) verrà visualizzato un ologramma condiviso che mostra i punti vita rimanenti.
        \end{itemize}
    \subsection{Requisiti non funzionali}\label{subsec:requisitiNonFunzionali}
        \begin{itemize}
            \item L'applicazione dovrà sfruttare la tecnologia WebXR per rendere fruibile, tramite un qualsiasi browser compatibile, l'esperienza di gioco.
        \end{itemize}

\section{Design architetturale}\label{sec:design}
In questo paragrafo si scenderà nel dettaglio dell'architettura del sistema, descrivendo le scelte progettuali effettuate per il corretto funzionamento delle tecnologie adottate.\\
\newline
Il framework di base da cui si è sviluppato il progetto è Croquet. La documentazione cita che per una buona realizzazione di un progetto Croquet bisogna cercare di mantenere una 
struttura speculare tra model e view. Nel primo vengono immagazzinate le informazioni e nella seconda vengono visualizzati i dati. Il \texttt{Croquet.Model} dovrà essere 
completamente indipendente dal framework di visualizzazione, in modo da poter essere utilizzato in qualsiasi contesto. Al contrario, nelle
\texttt{Croquet.View} si dovranno trovare solo componenti visibili all'utente, senza alcuna logica di business.\\
Croquet gestisce i dati condivisi creando una loro istanza all'interno di ogni client e sincronizzandoli tramite l'utilizzo dei reflector. Per comprendere meglio il funzionamento
dei passaggi successivi, si consiglia di immaginare questi dati come all'interno di un server fittizio a cui ogni client ha accesso come mostrato in figura~\ref{fig:CroquetServer.png}.\\
\img{CroquetServer.png}{In alto la vera struttura di Croquet, in basso una rappresentazione semplificata.}
Al contrario del model, ogni view va vista come una istanza di realtà aumentata unica per ogni utente. Ogni giocatore è immerso nella propria simulazione WebXR con un proprio 
\textit{engine}, una priopria telecamera e i propri oggetti di scena. Queste simulazioni interagiscono tra loro grazie agli eventi: all'interazione dell'utente con gli oggetti di scena,
la view invia un evento al model, che lo elabora, modifica il proprio stato e aggiorna le view degli altri utenti come mostrato in figura~\ref{fig:MultiUserBABYLON.png}.\\
\img[1]{MultiUserBABYLON.png}{Ogni utente ha la propria istanza di BabylonJS.}
\newline
Un ultimo aspetto dell'architettura da considerare è la creazione di una API che adattasse le classi \textit{general-purpose} di Croquet alle specifiche del progetto. Queste classi si pongono
in mezzo tra i model e le view utilizzate nell'elaborato e quelle di Croquet, modificando e riadattando le \textit{feature} già presenti nel framework e aggiungendone di nuove. 
Nello specifico si è deciso di implementare una \texttt{BaseView} ed un \texttt{BaseModel} che ereditassero rispettivamente da \texttt{Croquet.View} e \texttt{Croquet.Model}, dai 
quali poi estenderanno tutte le altre viste e i modelli del progetto.\\
Il \texttt{BaseModel} riporta le seguenti \textit{feature}:
\begin{itemize}
    \item \textbf{automatizzazione della parentela}: ogni modello creato con queste classe avrà un riferimento al modello padre;
    \item \textbf{creazione di un log personalizzato}: ogni modello avrà un log personalizzato con il proprio nome, utile per il debug;
    \item \textbf{gestione della distruzione}: ogni modello ascolterà l'evento \textit{`game-over'} avente come \textit{scope} il \textit{session ID} di modo che, in qualunque momento, in qualsiasi
    punto del codice, se il componente valuta che la partita sia terminata, può lanciare questo evento e distruggere tutti i modelli;
    \item \textbf{nuova inizializzazione}: ogni modello disporrà di due metodi da definire, \texttt{\_initialize} che fa le veci del vecchio \texttt{init}, ora utilizzato per il funzionamento
    del \texttt{BaseModel}, e \texttt{\_subscribeAll} in cui vanno inserite tutte le sottoscrizioni che quel modello deve effettuare.
\end{itemize}
Si noti che la separazione tra \texttt{\_initialize} e \texttt{\_subscribeAll} è solo a scopo di leggibilità del codice. Queste funzioni vengono chiamate entrambe all'avvio del modello, per
tanto non risulterebbe un problema se si facesse una sottoscrizione in \texttt{\_intialize} piuttosto che una inizializzazione di variabile in \texttt{\_subscribeAll}.\\
Si riporta il codice della classe \texttt{BaseModel} nel listato~\ref{lst:BaseModel.js}.
\code{BaseModel.js}{Classe \texttt{BaseModel}.}

Prima di procedere con l'elenco delle funzionalità della \texttt{BaseView} si vuole sottolineare che quando si parlerà di molteplici view, non si farà riferimento ad esse in senso
\textit{orizzontale}, ovvero che ogni view contraddistingua un utente diverso, bensì in senso \textit{verticale}, ovvero che ogni utente disponga di più view, ognuna delle quali mostra
oggetti diversi (si faccia riferimento alla figura\ref{fig:MultiViews.png}).\\
\img[0.6]{MultiViews.png}{Visione \textit{verticale} e \textit{orizzontale} delle view a confronto.}
Le feature che presenta la \texttt{BaseView} sono:
\begin{itemize}
    \item \textbf{automatizzazione della parentela}: ogni view creata con questa classe avrà un riferimento alla view padre;
    \item \textbf{automatizzazione del riferimento al model}: ogni vista creata con questa classe avrà un riferimento al model corrispettivo;
    \item \textbf{creazione di un log personalizzato}: ogni view avrà un log personalizzato con il proprio nome, utile per il debug;
    \item \textbf{nuova inizializzazione}: ogni view disporrà di tre metodi da definire, \texttt{\_initialize} che fa le veci del vecchio \texttt{init}, ora utilizzato per il funzionamento
    della \texttt{BaseView}, \texttt{\_subscribeAll} in cui vanno inserite tutte le sottoscrizioni che quella view deve effettuare e \texttt{\_initializeScene} che deve contenere tutte le
    inizializzazioni relative ad oggetti di scena;
    \item \textbf{automatizzazione dell'aggiornamento}: un punto cieco dell'architettura di Croquet è che la chiamata del metodo \texttt{update} non viene perpetrata da una view all'altra.
    Fornendo una lista \texttt{children} riempita con i riferimenti a tutte le view figlie, la \texttt{BaseView} potrà accedere a tutte le viste dalle quali poi richiamare i rispettivi
    \texttt{update} così da sopperire alla mancanza di Croquet. Viene fornito anche un metodo \texttt{\_update} da sovrascrivere per aggiungere funzionalità all'aggiornamento;
    \item \textbf{gestione della distruzione}: ogni view ascolterà l'evento \textit{`game-over'} avente come \textit{scope} il \textit{session ID} di modo che, se venisse lanciato, la view
    si distrugga automaticamente. Per attuare ciò viene lasciata una lista \texttt{sceneObjects} da riempire con tutte le \textit{mesh} di BabylonJS istanziate in modo tale che,
    alla chiamata di distruzione, la \texttt{BaseView} possa distruggere anch'essi. Inoltre, essendo una UI, si è previsto che le view avessero bisogno di un tempo d'attesa prima di 
    cancellare tutta la scena, affinchè ogni oggetto faccia la sua uscita di scena e/o mostri informazioni riguardanti il termine del gioco. Per realizzare ciò si è costruito un metodo 
    da estendere chiamato \texttt{\_endScene} nel quale effettuare tutte le animazioni del caso e che restituisca il numero di millisecondi da aspettare prima di lanciare il comando di 
    \texttt{detach};
    \item \textbf{informazioni condivise}: si è creata una struttura dati che fosse accessibile da tutte le viste e che racchiudesse le informazioni univoche per ogni partita come una sorta
    di \textit{singleton pattern}. Qui si possono trovare informazioni come il riferimento all'\textit{engine} di BabylonJS utilizzato, alla scena o alla telecamera.
\end{itemize}
Si noti che, per quanto possa essere richiamata da qualsiasi punto del codice, la \textit{`game-over'} venga lanciata solo dal modello. Questo perchè, essendo il modello l'unico 
componente che conosce lo stato della partita, è l'unico che può valutare se la partita sia terminata o meno.\\
Anche in questo caso, si lascia il sorgente della classe \texttt{BaseView} nel listato~\ref{lst:BaseView.js}.
\code{BaseView.js}{Classe \texttt{BaseView}.}
Fino ad ora si sono esposte le scelte progettuali effettuate per la realizzazione del progetto. Nella prossima sezione si vedranno più nel dettaglio le funzionalità implementate
nel progetto.

\section{Progettazione dettagliata}\label{sec:progettazione}
In quest'ultima sezione si affronteranno le strutture del progetto nello specifico, analizzando le classi e le loro funzionalità. La trattazione verrà affrontata in ordine di 
creazione delle classi, in modo da collegare questa discussione al \textit{gameplay} del progetto.\\
\newline
Prima di iniziare con la descrizione delle classi, si vuole esporre dei punti chiave comuni lungo tutto il progetto.\\
Seguendo la logica del \textit{single-responsibility principle}\footnote{Single responsibility principle: principio di progettazione software che afferma che ogni modulo o classe deve
essere responsabile di una singola funzionalità.}, si è strutturata una rete di \texttt{Croquet.Model} che distribuisse ogni funzionalità del dominio ad un modello diverso, in modo da tenere separati i compiti e mantenere una
gerarchia ordinata e intuitiva.\\
Altro tassello importante è la decisione di non utilizzare ereditarietà, preferendo la composizione in quanto ritenuta più flessibile e meno vincolante. La struttura si può
raffigurare comunque in una relazione padre-figlio in cui però non è il figlio a estendere il padre, bensì il padre a creare e contenere il figlio.\\
In ultima istanza si sono create delle classi JavaScript ausiliarie all'architettura di Croquet che fungono da supporto per la gestione dei dati. Queste classi sono state create per 
racchiudere al loro interno le caratteristiche più importanti dei concetti da modellare e quindi facilitare la manutenzione e migliorare la leggibilità del codice.\\

\subsection{Root}\label{subsec:rootclass}
Il \texttt{RootModel} e la \texttt{RootView} sono le prime classi create nel progetto. Queste classi sono state create con la funzione di modellare una scena di base da cui far
partitre l'utente, per creare una separazione tra il collegamento alla sessione e l'inzio del gioco vero e proprio. L'obiettivo era infatti quello di rendere la gestione delle
connessioni più semplice e logico.\\
Il \texttt{RootModel} nello specifico tiene traccia di tutte le view che si collegano alla sessione, in modo da creare un'istanza del gioco all'avvento della prima view collegata e 
distruggere il gioco alla disconnessione dell'utlima view. Fornisce anche il metodo \texttt{restart} che consente di riavviare la simulazione in caso di fine gioco ma ancora utenti 
connessi.\\
La classe \texttt{RootView} si occupa di creare la scena di base, ovvero quella che l'utente vede prima di iniziare il gioco. Questa consiste semplicemente di un bottone che, se 
premuto, avvia la simulazione vera e propria. Seppur possa sembrare abbasta inutile, la parte più importante di questa classe risiede nel \textit{dietro alle quinte} della scena. 
Qui infatti si crea e avvia un \textit{rendering-loop}, ovvero un ciclo infinito che aggiorna la scena ad ogni frame, l'\textit{engine}, la scena, la telecamera, la luce e tutta
una serie di funzionalità dedite al corretto avvio di una scena XR (se è stato rilevato un dispositivo affine).\\
Queste due classi sono le uniche in tutta la \textit{codebase} a non autodistruggersi all'avvento del `game over'. Se così non fosse, al termine della partita l'utente si ritroverebbe
con una schermata vuota e dovrebbe ricaricare la pagina per avviare una nuova sessione. Per aggirare il problema si è deciso di cambiare il comportamento che hanno queste classi al 
termine della partita, facendo sì che ricreino la scena (e anche i dati) da zero.\\

\subsection{Game}\label{subsec:game}
\subsection{Player}\label{subsec:player}
\subsection{LifePoints}\label{subsec:lifepoints}
\subsection{Hand}\label{subsec:hand}
\subsection{Turn}\label{subsec:turn}
\subsection{Battlefield}\label{subsec:battlefield}

Cose da dire:
\begin{itemize}
\item Definiti questi princìpi cardine si sono quindi costruite le fondamenta su cui basare il video game. La classe più importante, da cui poi si sviluppa tutta la struttura, è 
\texttt{GameModel}. Questa classe di fatto non gestisce una struttura dati ma fa da contenitore per tutti i modelli del gioco. Qui si trovano i riferimenti ai modelli dei giocatori,
del turno e del campo di battaglia.\\
Altra funzionalità importante per questa classe è di gestire le connessioni e i ruoli. All'avvio, in base al numero di partecipanti già presenti, questa classe avrà il compito di 
assegnare un ruolo al nuovo utente connesso scegliendo tra \textit{player 1}, \textit{player 2} e \textit{spettatore}. Inoltre, se un utente dovesse disconnettersi, questa classe
dovrà gestire una sua possibile riconnessione come anche prevedere una sequenza di terminazione nel caso in cui l'utente non dovesse riconnettersi.\\
Nella controparte \texttt{GameView} si possono trovare gli stessi riferimenti alle view corrispondenti dei modelli citati. Si noti che, dato che il model non contiene una struttura
dati, la \texttt{GameView} non crea alcun componente visibile all'utente, mantenendo coerenza con il principio di specularità tra model e view.\\
\end{itemize}

\chapter*{Conclusioni}
\addcontentsline{toc}{chapter}{Conclusioni}
\markboth{CONCLUSIONI}{CONCLUSIONI}
I concetti fondamentali su cui si basa questo elaborato sono tre: XR, il video game e le applicazioni distribuite. Il primo rappresenta la novità, l'aspetto di ricerca che persegue questa
tesi. Il secondo è il contesto in cui si è deciso di applicare la tecnologia XR che, essendo un gioco, rende più interessante e apprezzabile il progetto. Il terzo raffigura 
la sfida che si propone la tesi, ovvero collegare tramite una rete distribuita due o più utenti in modo da poter giocare insieme.\\
Ognuno di questi elementi ha portato con sè un tema da sviluppare e sviscerare lungo tutta la tesi. In queste righe si cerca di esporre le principali difficoltà che hanno comportato queste
tre tematiche e come si è cercato di affrontarle. Si chiuderà con l'esposizione di possibili sviluppi futuri per questo elaborato.\\
\newline
La combinazione delle varie tecnologie all'interno di un singolo progetto è considerabile un risultato di per sè. La sfida di integrare un sistema di distribuzione
come Croquet con un sistema di simulazione come WebXR ha portato a problematiche complesse e mai affrontate prima. Ancora prima della loro integrazione, c'è stata tutta una fase 
dedicata allo studio dei framework e alla scelta di quale fosse il più adatto per il progetto. Dopodichè è stato necessario esplorare operativamente le infrastrutture scelte 
applicandole ad un prototipo del sistema. Infine, si è passati alla fase di integrazione vera e propria, che ha portato, tra le altre cose, problemi di comunicazione tra i vari 
componenti e di sincronizzazione tra i vari utenti.\\
Si vuole espandere quest'ultima problematica relativa alla sincronizzazione in quanto è stata una delle più complesse e che ha richiesto più tempo per essere risolta. Per venirne a 
capo sono state create strutture \textit{ad-hoc} come la \texttt{Root} e la \texttt{Game} in modo da poter gestire al meglio ogni caso di connessione, disconnessione e riconnessione 
che gli utenti potessero fare.\\
Un'ultima sfida che si è affrontata è stata cercare di rendere il sistema il più generico ed esetendibile possibile. Per quanto si sia cercato di lavorare in tal senso, tutt'ora vi sono
componenti, metodi e interazioni che avrebbero bisogno di essere riorganizzati e ristrutturati per rendere il progetto più modulare. Al netto di questo, il software fornisce comunque
dei pattern di base che possono essere utilizzati per creare applicazioni in realtà aumentata distribuite.\\
\newline
I piani futuri per questo elaborato sono molteplici. Si potrebbe pensare ad una standardizzazione delle classi di model e view e delle loro comunicazioni, in modo da poter modellare
un framework che permetta di creare applicazioni di realtà aumentata distribuite in maniera più semplice. Si potrebbero revisionare ed espandere la struttura di \texttt{BaseModel} e
\texttt{BaseView}, in modo da fornire un'interfaccia più completa e più semplice da utilizzare.\\
Si è pensato anche alla continuazione del progetto per la conclusione delle \textit{feature} secondarie che lo riguardano come gli effetti delle carte, la possibilità di giocare
mostri tramite sequenze complicate tipo le \textit{evocazioni}\footnote{Evocazione: nel gioco di carte questo termine è sinonimo di giocare sul campo un mostro.} speciali e le evocazioni 
\textit{Xyz}, e la possibilità di giocare magie \textit{terreno}\footnote{Magia terreno: carta magia che nella serie animata modificavano l'ambiente circostante.} che, nel contesto di
un applicativo XR, riscontrerebbero ancora più successo.\\	
\chapter*{Ringraziamenti}
\addcontentsline{toc}{chapter}{Ringraziamenti}
\markboth{RINGRAZIAMENTI}{RINGRAZIAMENTI}

Desidero esprimere la mia sincera gratitudine a tutte le persone che hanno contribuito in modo significativo al completamento di questa tesi di laurea. Il mio percorso di studio è 
stato arricchito dalla generosità, dal supporto e dalla fiducia di molti individui, ai quali vorrei dedicare questa sezione di ringraziamenti.\\
\newline
Innanzitutto, vorrei ringraziare il Professore Alessandro Ricci per avermi concesso l'opportunità di svolgere questa tesi sotto la sua guida esperta. La sua competenza e il suo 
sostegno costante sono stati fondamentali per la realizzazione di questo lavoro.\\
\newline
Un sentito ringraziamento va anche al corelatore della tesi, il Dottore Samuele Burattini, che ha dedicato tempo ed energie per seguirmi passo dopo passo nell'intero processo di
sviluppo della tesi. La sua pazienza e la sua disponibilità sono state inestimabili per il mio successo.\\
\newline
La mia famiglia merita un profondo riconoscimento per il loro costante appoggio e la fiducia incondizionata che mi hanno offerto durante il mio percorso di studi.
Senza il loro amore e il loro incoraggiamento, questo traguardo sarebbe stato molto più difficile da raggiungere.\\
\newline
Un ringraziamento speciale va ai miei amici universitari del gruppo \textit{Predapio}, con i quali ho condiviso gioie, sfide e momenti di studio. La vostra compagnia ha reso 
l'esperienza universitaria unica e indimenticabile.\\
\newline
Ai miei amici al di fuori dell'università, desidero esprimere la mia gratitudine per avermi sostenuto in ogni fase della mia vita accademica. La vostra amicizia è stata una fonte
di gioia e ispirazione.\\
\newline
Un grazie speciale va alla mia morosa Silvia, che ha sempre ascoltato pazientemente le mie discussioni sull'argomento della tesi, offrendo supporto emotivo e incoraggiamento
quando ne avevo più bisogno. La tua comprensione è stata un motore fondamentale per il mio successo.\\
\newline
In conclusione, questa tesi rappresenta un capitolo importante nella mia vita, e l'aiuto e il supporto di tutte queste persone sono stati fondamentali per raggiungere questo traguardo. 
Grazie di cuore a ciascuno di voi.






	
\backmatter	

\bibliography{bibliografia}{}
\bibliographystyle{plain}

\end{document}