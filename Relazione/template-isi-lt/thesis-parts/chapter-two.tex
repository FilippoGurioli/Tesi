\chapter{Capitolo con Immagini e Codice}

Introduzione al capitolo

\section{Prima sezione del secondo capitolo}

Qui di seguito un'immagine. Tutte le immagini devono essere inserite nella cartella \texttt{images}

\begin{figure}[H]
\centering
\includegraphics[scale=0.8]{images/placeholder.jpg}
\caption{Testo della didascalia}
\label{fig:figure1}
\end{figure}

Qui invece il testo con il riferimento all'immagine: Figura \ref{fig:figure1}.

Qui un'immagine con due sotto-immagini:

\begin{figure}[H]
\centering
\begin{subfigure}[b]{0.42\textwidth}
\includegraphics[width=\textwidth]{images/placeholder.jpg}
\caption{Didascalia prima figura}
\label{fig:figure2}
\end{subfigure}
\qquad
\begin{subfigure}[b]{0.42\textwidth}
\includegraphics[width=\textwidth]{images/placeholder.jpg}
\caption{Didascalia seconda figura}
\label{fig:figure3}
\end{subfigure}
\caption{Didascalia globale}
\label{fig:figures12}
\end{figure}

Altri tipi di riferimenti: Figura \ref{fig:figures12} oppure \ref{fig:figure3}.

\section{Seconda sezione con codice sorgente}

Il codice sorgente può essere inserito mediante un link al file del codice sorgente, in questo caso il file con il codice va inserito nella cartella \texttt{code}.

\lstinputlisting[caption={Didascalia o il nome del file}, label=lst:listato1]{code/Main.java}


