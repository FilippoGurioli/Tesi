\chapter{Capitolo d'esempio}
Qui una breve descrizione del contenuto del capitolo o, eventualmente, testo introduttivo alle sezioni che seguono

\paragraph{NOTA BENE:} In Latex per andare a capo tra un paragrafo e l'altro del testo si deve lasciare una riga vuota nel sorgente del testo. In questo modo la prima parola di ogni paragrafo sarà centrata. Non è necessario utilizzare il doppio backslash per andaer a capo forzatamente.


\section{Prima Sezione}
Qui il testo della prima sezione. Questa è  una parola in \textit{corsivo}, questa invece è una parola in \textbf{grassetto} e questa è una parola in \texttt{monotype}.

\begin{center}
Questo è un paragrafo centrato!
\end{center}

Dimensione del testo:\\

\LARGE{Testo} \Large{Testo} \large{Testo} \normalsize{Testo} \small{Testo} \footnotesize{Testo}\\

Qui una citazione bibliografica \cite{Lingiardi}

Qui un link \url{www.google.it}


\subsection{Prima Sottosezione}

Qui l'esempio di un elenco puntato:

\begin{itemize}
\item Primo elemento,
\item Secondo elemento,
\item Terzo elemento.
\end{itemize}

Qui l'esempio di un elenco con un sotto-elenco:

\begin{itemize}
\item Primo elemento,
\begin{itemize}
\item Primo elemento del sotto-elenco
\item Secondo elemento del sotto-elenco
\end{itemize}
\item Secondo elemento,
\item Terzo elemento.
\end{itemize}

Qui infine l'esempio di un elenco numerato

\begin{enumerate}
\item Testo
\item Testo
\item Testo
\end{enumerate}

\section{Seconda Sezione}\label{sec:Sezione2}

Qui l'esempio di una formula matematica:

$ G = \gamma\dfrac{m_{1}m_{2}}{r^{2}} $

\section{Terza Sezione}

Qui il riferimento ad una sezione utilizzando una label: Sezione \ref{sec:Sezione2}

Qui la definizione di due paragrafi con titoli.

\paragraph{Titolo 1} testo del paragrafo

\paragraph{Titolo 2} testo del paragrafo, con una nota a piè di pagina\footnote{Questo è il testo della nota, in cui si può \textit{utilizzare} qualunque stile di \textbf{formattazione}}

