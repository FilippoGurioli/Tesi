\chapter{Stato dell'arte}\label{chap:Letteratura}
Di seguito si riporteranno informazioni circa lo stato dell'arte riguardante la Extended Reality, 
in particolare, si toccherà il tema dello sviluppo di videogiochi in ambito Augmented Reality e Virtual Reality. 
Successivamente si analizzerà la letteratura riguardante il tema della cooperativa, come questa sia attualmente 
realizzata nei software comuni e come invece viene applicata al caso di studio di questa tesi, 
sottolineando l'importanza del web al giorno d'oggi.

\section{Extended Reality}\label{sec:XR}
    La Extended Reality è un campo molto vasto che comprende diverse tecnologie ed è in continua 
    evoluzione. Per questo motivo, non è possibile fornire una definizione univoca e completa.
    Lo scopo della sezione seguente è quindi quello di chiarire i concetti fondamentali riguardanti questa 
    tecnologia, esplorare le diverse accezioni e presentare alcuni esempi di utilizzo.
    \subsection{Definizione}\label{subsec:XRDef}
        Con il termine Extended Reality (abbreviato in XR) si intende un insieme di tecnologie che 
        permettono di estendere la realtà, ovvero di aggiungere informazioni al mondo reale. Sotto 
        questa macro definizione rientrano le tecnologie di realtà aumentata (AR), realtà virtuale
        (VR) e realtà mista (MR). Per AR si intende la tecnologia che permette di sovrapporre
        informazioni digitali al mondo reale, mentre per VR si intende la tecnologia che permette
        di immergere l'utente in un ambiente virtuale. La MR è una tecnologia che permette di
        conciliare il mondo reale con quello virtuale in modo totalmente trasparente per l'utente. Un aspetto 
        molto interessante riguardante queste tecnologie sono i dispositivi di input e output. Venendo
        completamente immersi in queste realtà virtuali, l'utilizzo di una tastiera o un mouse risulterebbe 
        inefficace, si sono quindi progettati dispositivi ad hoc per l'interazione (input) e per
        la ricezione delle informazioni (output). Per quanto riguarda l'input, si possono utilizzare
        dei controller, dei guanti o il proprio corpo, mentre per l'output domina l'utilizzo di
        visori. Questi dispositivi sono in grado di rilevare i movimenti dell'utente e di trasmetterli 
        al sistema, che li elabora e li utilizza per modificare l'ambiente virtuale. Questo ambiente virtuale 
        può essere visualizzato dall'utente tramite un visore, che può essere un visore VR, un visore
        AR o un visore MR.
        \paragraph{Augmented Reality e Mixed Reality} sono due concetti molto simili, dei quali si trovano
            definizioni ambigue, talvolta discordanti. La definizione più comunemente accettata per Augmented 
            Reality è quella fornita da Azuma\cite{Azuma1997}:
            \begin{quote}
                \textit{AR systems have the following three characteristics: (1) combine real and virtual, 
                (2) are interactive in real time, and (3) are registered in 3D.}
            \end{quote}
            Si noti che non è specificato nessun tipo di dispositivo di output, potrebbero essere 
            AR anche dispositivi che forniscono informazioni tattili, gustative o olfattive, benchè
            la loro implementazione è di fatto ancora solo un'idea (motivo per il quale nella sezione
            precedente non li si è menzionati). La definizione formale di Mixed Reality è invece 
            fornita da Milgram\cite{Milgram1994}:
            \begin{quote}
                \textit{A mixed reality (MR) system is one that combines real and virtual environments 
                seamlessly.}
            \end{quote}
            Questa definizione è molto simile a quella di Azuma, ma non è specificato il fatto che
            l'ambiente virtuale debba essere tridimensionale. La definizione di Milgram
            non specifica, inoltre, che l'ambiente virtuale debba essere sovrapposto a quello reale, ma che
            i due ambienti debbano essere combinati in modo da risultare indistinguibili. In quest'ottica 
            si può quindi affermare che la AR è un caso particolare di MR, in cui l'ambiente virtuale è
            sovrapposto a quello reale. \\
            Il libro di Schmalstieg e Höllerer \textit{Augmented Reality: Principles and Practice}\cite{Schmalstieg2016} 
            fornisce una schematizzazione della gerarchia tra AR e MR mostrata in figura\ref{fig:ARvsMR}
            che permette di comprendere meglio la relazione tra le due tecnologie. 
            \img{ARvsMR}{Gerarchia tra AR e MR - \textit{`The mixed reality continuum'}.}
            
    \subsection{Applicazioni}\label{subsec:XRTipologie}
        

\section{Videogiochi}\label{sec:Videogiochi}
Qui verrà introdotto Unity.
    \subsection{Definizione}\label{subsec:VideogiochiDef}
    \subsection{Applicazioni}\label{subsec:VideogiochiTipologie}

\section{Cooperativa}\label{sec:Coop}
    \subsection{Definizione}\label{subsec:CoopDef}
    \subsection{Tipologie}\label{subsec:CoopTipologie}
    \subsection{Applicazioni}\label{subsec:CoopApplicazioni}
        \paragraph{Al giorno d'oggi}\label{par:CoopOggi}
        \paragraph{Videogiochi}\label{par:CoopVideogiochi}
        \paragraph{Web}\label{par:CoopWeb}
        Qui verrà fatto il confronto tra Unity e WebXR.
