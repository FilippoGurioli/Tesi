\chapter{Stato dell'arte}\label{chap:Letteratura}
Di seguito si riporteranno informazioni circa lo stato dell'arte riguardante la extended reality, 
in particolare, si toccherà il tema di dello sviluppo di videogiochi in ambito AR e VR. Successivamente
si analizzerà la letteratura riguardante il tema della cooperativa, come questa sia attualmente 
realizzata nei software comuni e come invece viene applicata al caso di studio di questa tesi, 
sottolineando l'importanza del web al giorno d'oggi.

\section{Extended Reality}\label{sec:XR}
    \subsection{Definizione}\label{subsec:XRDef}
    \subsection{Tipologie}\label{subsec:XRTipologie}

\section{Videogiochi}\label{sec:Videogiochi}
Qui verrà introdotto Unity.
    \subsection{Definizione}\label{subsec:VideogiochiDef}
    \subsection{Tipologie}\label{subsec:VideogiochiTipologie}

\section{Cooperativa}\label{sec:Coop}
    \subsection{Definizione}\label{subsec:CoopDef}
    \subsection{Tipologie}\label{subsec:CoopTipologie}
    \subsection{Applicazioni}\label{subsec:CoopApplicazioni}
        \paragraph{Al giorno d'oggi}\label{par:CoopOggi}
        \paragraph{Videogiochi}\label{par:CoopVideogiochi}
        \paragraph{Web}\label{par:CoopWeb}
        Qui verrà fatto il confronto tra Unity e WebXR.
