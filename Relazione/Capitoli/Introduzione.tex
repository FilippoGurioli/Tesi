\chapter{Introduzione}\label{chap:Introduzione}
\markboth{INTRODUZIONE}{INTRODUZIONE}

I videogiochi sono sempre stati una componente fondamentale che accompagna lo 
sviluppo teconologico, a partire da uno dei primi computer della storia, il DEC PDP-1,
che nel 1961 ospitava il primo videogioco della storia, \textit{''Spacewar!''}, fino ad arrivare
ai giorni nostri, in cui il settore videoludico si è espanso in ogni sorgente tecnologica
che ci circonda, dallo smartphone, al computer, fino alle console. Da quello che si 
potrebbe definire uno svago dei programmatori è nata una disciplina a tutti gli effetti,
suddivisa su varie tematiche: dal game design, al game programming, fino alla computer
graphics. Per ognuna di queste branche la letteratura è vastissima e in continua
evoluzione. Questa tesi si pone l'obiettivo di estendere la letteratura riguardante
il game programming, ovvero quella disciplina che si occupa di gestire la logica
del modello di gioco, in particolare, si vuole esplorare un nuovo campo di ricerca
che sta prendendo sempre più piede: la realtà aumentata.\\\newline
L'arrivo di nuove tecnologie ha sempre portato a nuovi modi di giocare, e con la 
realtà aumentata non è stato da meno. La realtà aumentata è una tecnologia che 
permette di sovrapporre elementi della realtà che ci circonda con elementi virtuali, 
creando un'esperienza ibrida, unica nel suo genere. (TODO:Aggiungere una microstoria della realtà aumentata)\\\newline
Come ogni grande scoperta nell'ambito informatico, anche l'arrivo della realtà 
aumentata ha portato nuove sfide per i programmatori, in particolare, un campo di ricerca tutt'ora molto attivo è la condivisione 
dell'esperienza. (TODO:approfondire - primi giochi condivi, soluzioni attuali, come lo si gestisce con realtà aumentata attualmente ...)\\\newline
È da queste premesse che nasce il progetto di tesi, nelle prossime
pagine si vuole infatti definire pattern noti e best practice per lo sviluppo di 
videogiochi in realtà aumentata cooperativa tramite l'analisi di un caso di studio
particolare: il gioco di carte \textit{''Yu-Gi-Oh!''}.\\
\\
