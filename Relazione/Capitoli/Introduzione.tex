\chapter{Introduzione} \label{chap:Introduzione}
I videogiochi sono sempre stati una componente fondamentale che accompagna lo 
sviluppo teconologico, a partire da uno dei primi computer della storia, il DEC PDP-1,
che nel 1961 ospitava il primo videogioco della storia, Spacewar!, fino ad arrivare
ai giorni nostri, in cui il settore videoludico si è espanso in ogni sorgente tecnologica
che ci circonda, dallo smartphone, al computer, fino alle console.\\
L'arrivo di nuove tecnologie ha sempre portato a nuovi modi di giocare, e con la 
realtà aumentata non è stato da meno. La realtà aumentata è una tecnologia che permette di sovrapporre elementi della
realtà che ci circonda con elementi virtuali, creando un'esperienza ibrida, unica
nel suo genere.\\
Come ogni grande scoperta, anche l'arrivo della realtà aumentata ha portato nuove
sfide per i programmatori, in particolare, nella scena videoludica e non, un campo 
di ricerca tutt'ora molto attivo è la condivisione dell'esperienza.\\
È da queste premesse che nasce il progetto di tesi, nelle prossime
pagine si vuole infatti definire pattern noti e best practice per lo sviluppo di 
videogiochi in realtà aumentata cooperativa tramite l'analisi di un caso di studio
particolare: il gioco di carte Yu-Gi-Oh!.\\
\\
