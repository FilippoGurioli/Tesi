\chapter{Introduzione}\label{chap:Introduzione}
\markboth{INTRODUZIONE}{INTRODUZIONE}

% I videogiochi sono sempre stati una componente fondamentale che accompagna lo 
% sviluppo teconologico, a partire da uno dei primi computer della storia, il DEC PDP-1,
% che nel 1961 ospitava il primo videogioco della storia, \textit{''Spacewar!''}. Arrivando
% ai giorni nostri, il settore videoludico si è espanso in ogni sorgente tecnologica
% che ci circonda, dallo smartphone, al computer, fino alle console. Da quello che si 
% potrebbe definire uno svago dei programmatori è nata una disciplina a tutti gli effetti,
% suddivisa su varie tematiche: dal game design, al game programming, fino alla computer
% graphics. Per ognuna di queste branche la letteratura è vastissima e in continua
% evoluzione. Questa tesi si pone l'obiettivo di estendere la letteratura riguardante
% il game programming, ovvero quella disciplina che si occupa di gestire la logica
% del modello di gioco, in particolare, si vuole esplorare un nuovo campo di ricerca
% che sta prendendo sempre più piede: la realtà aumentata.\\
% \newline
% L'arrivo di nuove tecnologie ha sempre portato a nuovi modi di giocare e con la 
% realtà aumentata non è stato da meno. La realtà aumentata è una tecnologia che 
% permette di sovrapporre elementi della realtà che ci circonda con elementi virtuali, 
% creando un'esperienza ibrida, unica nel suo genere. Il primo visore 
% AR è stato inventato negli anni '60, si chiama \textit{`Spada di Damocle'} e non è altro 
% che un casco con un piccolo schermo rudimentale che mostrava immagini semplici 
% tridimensionali. Nella sua semplicità questa scoperta porterà poi ai famosi Google 
% Glass fino ad arrivare agli Hololens, tecnologia al centro di questo progetto di tesi.\\
% \newline
% Come ogni grande scoperta nell'ambito informatico, anche l'arrivo della realtà 
% aumentata ha portato nuove sfide per i programmatori, in particolare, un campo 
% di ricerca tutt'ora molto attivo è la condivisione dell'esperienza. Partendo dal 
% primo gioco multiplayer mai creato, \textit{Tennis for Two}, in cui si giocava dalla
% stessa macchina, si è poi passati ai primi giochi online di cui il padre fondatore
% è \textit{Doom}, ancora in LAN, fino ad arrivare ai grandi giochi distribuiti come
% \textit{Minecraft} o \textit{Fortnite}. Tra questi bisogna anche menzionare la linea
% \textit{Call Of Duty} che rappresenta l'esempio più famoso di gioco multiplayer online
% con connessione peer-to-peer. In questo contesto sono nate anche le prime idee per
% connettere gli utenti in realtà virtuale e/o aumentata, trovando soluzioni sempre
% più creative per risolvere i problemi di comunicazione tra i giocatori.\\
% \newline
% È da queste premesse che nasce questo progetto di tesi, nelle prossime
% pagine si vogliono definire infatti pattern noti e best practice per lo sviluppo di 
% videogiochi in realtà aumentata cooperativa tramite l'analisi di un caso di studio
% particolare: il gioco di carte \textit{''Yu-Gi-Oh!''}.\\
